The program
\begin{listing}{1}
  int x = 0;
  int y = 10;

  /*@ loop invariant 0 <= x < 11;
    @*/
  while (y > 0) {
    x++;
    y--;
  }
\end{listing}
is not correctly annotated, even if it is true that \lstinline|x|
remains smaller
than 11 during the execution. This is because it is not true that the
property \lstinline|x<11| is preserved by the execution of
\lstinline|x++ ; y--;|. A
correct loop invariant could be \lstinline|0 <= x < 11 && x+y == 10|. It holds
at loop entrance and is preserved (under the assumption of the loop
condition \lstinline|y>0|).

Similarly, the following general invariants are not inductive:
\begin{listing}{1}
  int x = 0;
  int y = 10;

  while (y > 0) {
    x++;
    //@ invariant 0 < x < 11;
    y--;
    //@ invariant 0 <= y < 10;
  }
\end{listing}
since \lstinline|0 <= y < 10| is not a consequence of
hypothesis \lstinline|0 < x < 11| after executing \verb|y--|; and
\lstinline|0 < x < 11| cannot be deduced from \lstinline|0 <= y < 10| after
looping back
through the condition \lstinline|y>0| and executing \lstinline|x++|. Correct
invariants could be:
\begin{listing}{1}
  while (y > 0) {
    x++;
    //@ invariant 0 < x < 11 && x+y == 11;
    y--;
    //@ invariant 0 <= y < 10 && x+y == 10;
  }
\end{listing}
