\subsection{Parsing annotations in practice}

In JML, parsing is done by just ignoring \verb|//@|, \verb|/*@| and
\verb|*/| and the level of lexing. This technique could modify the
semantics of the C code, for example: 
\begin{c}
return x /*@ +1 */ ;
\end{c}
In our language this is forbidden. Technically, the current
implementation of Frama-C isolates the comments in a first step of
syntax analysis, and then parses a second time. Nevertheless, the
grammar and the corresponding parser must be carefully designed to
avoid interaction of annotations with the code. For example, in such a
code: 
\begin{c}
  if (c) //@ assert P;
     c=1;
\end{c}
the statement \verb|c=1| must be understood as the \texttt{then}
branch of the \texttt{if}. This is ensured by the grammar below,
saying that \verb|assert| annotations are not statement themselves,
but attached to the statement that follows, like C labels.
