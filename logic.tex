\begin{syntax}
  logic_decl ::= logic_type_decl ;
          | logic_const_decl | logic_const_def ;
          | logic_predicate_decl | logic_predicate_def ;
          | logic_function_decl | logic_function_def ;
          | axiom_def
  \\
  logic_type_decl ::= "logic" "type" logic_type ( = logic_type_def)? ";" ; ``typedef'' au lieu de ``type'' ?
  \\
  logic_type ::= id | type_var id ;
                 | "(" type_var (, type_var)* ")" id 
                 \\
  type_var ::= "'" id 
  \\
  logic_type_def ::= TODO by Claude 
  \\
  logic_predicate_decl ::= "predicate" id fun_parameters ("reads" locations)? ";"
  \\
  logic_predicate_def ::= "predicate" id fun_parameters "{" pred "}"
  \\
  logic_function_decl ::= "logic" type_expr id fun_parameters ("reads" locations)? ";"
  \\
  logic_function_def ::= "logic" type_expr id fun_parameters "{" pred "}"
  \\
  type_expr ::= logic_type_expr | C_type_expr
  \\
  logic_type_expr ::= built_in_logic_type | id | type_expr id ;
                 | "(" type_expr (, type_expr)* ")" id 
                 | logic_type_expr ("*" logic_type_expr)+  ; product type 
  \\
  built_in_logic_type ::= "boolean" | "integer" | "real" 
  \\
  fun_parameters ::= "(" parameter (, parameter)* ")"
  \\
  parameter ::= type_expr | type_expr id | parameter "[]" | ... ; TODO: faire comme en C
  \\
  logic_const_decl ::= "logic" type_expr id 
  \\
  logic_const_def ::= "logic" type_expr id "=" term
  \\
axiom_def ::= "axiom" id ":" pred ";"
\end{syntax}