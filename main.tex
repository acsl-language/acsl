%; whizzy section -pdf -initex "pdflatex -ini"
\documentclass[a4paper,11pt,twoside,openright,web]{frama-c-book}
\usepackage{hevea}
\usepackage{ifthen}

%%% Environnements dont le corps est suprim�, et
%%% commandes dont la d�finition est vide,
%%% lorsque PrintRemarks=false

\usepackage{comment}

\newcommand{\framac}{\textsc{Frama-C}\xspace}
\newcommand{\acsl}{\textsc{ACSL}\xspace}
\newcommand{\eacsl}{\textsc{E-ACSL}\xspace}
\newcommand{\C}{\textsc{C}\xspace}
\newcommand{\jml}{\textsc{JML}\xspace}

\newcommand{\nodiff}{\emph{No difference with \acsl.}}
\newcommand{\except}[1]{\emph{No difference with \acsl, but #1.}}
\newcommand{\limited}[1]{\emph{Limited to #1.}}
\newcommand{\absent}{\emph{No such feature in \eacsl.}}
\newcommand{\absentwhy}[1]{\emph{No such feature in \eacsl: #1.}}
\newcommand{\absentexperimental}{\emph{No such feature in \eacsl, since it is
    still experimental in \acsl.}}
\newcommand{\absentexcept}[1]{\emph{No such feature in \eacsl, but #1.}}
\newcommand{\difficultwhy}[2]{\emph{#1 is usually difficult to implement, since
    it requires #2. Thus you would not wonder if most tools do not support it
    (or support it partially).}}
\newcommand{\difficultswhy}[2]{\emph{#1 are usually difficult to implement,
    since they require #2. Thus you would not wonder if most tools do not
    support them (or support them partially).}}
\newcommand{\difficult}[1]{\emph{#1 is usually difficult to implement. Thus
    you would not wonder if most tools do not support it (or support
    it partially).}}
\newcommand{\difficults}[1]{\emph{#1 are usually difficult to implement. Thus
    you would not wonder if most tools do not support them (or support
    them partially).}}

\newcommand{\changeinsection}[2]{\textbf{Section \ref{sec:#1}:} #2.}

\newcommand{\todo}[1]{{\large \textbf{TODO: #1.}}}

\newcommand{\markdiff}[1]{{\color{blue}{#1}}}
\newenvironment{markdiffenv}[1][]{%
  \begin{changebar}%
  \markdiff\bgroup%
}%
{\egroup\end{changebar}}

% true = prints remarks for the ACSL working group.
% false = prints no remark for the distributed version of ASCL documents
\newboolean{PrintRemarks}
\setboolean{PrintRemarks}{false}

% true = prints marks signaling the state of the implementation
% false = prints only the ACSL definition, without remarks on implementation.
\newboolean{PrintImplementationRq}
\setboolean{PrintImplementationRq}{true}

% true = remarks about the current state of implementation in Frama-C
% are in color.
% false = they are rendered with an underline
\newboolean{ColorImplementationRq}
\setboolean{ColorImplementationRq}{true}

%% \ifthenelse{\boolean{PrintRemarks}}%
%%       {\newenvironment{todo}{%
%%             \begin{quote}%
%%             \begin{tabular}{||p{0.8\textwidth}}TODO~:\itshape}%
%%            {\end{tabular}\end{quote}}}%
%%       {\excludecomment{todo}}

\ifthenelse{\boolean{PrintRemarks}}%
      {\newenvironment{remark}[1]{%
             \begin{quote}\itshape%
             \begin{tabular}{||p{0.8\textwidth}}Remarque de {#1}~:}%
           {\end{tabular}\end{quote}}}%
      {\excludecomment{remark}}

\newcommand{\oldremark}[2]{%
\ifthenelse{\boolean{PrintRemarks}}{%
            %\begin{quote}\itshape%
            %\begin{tabular}{||p{0.8\textwidth}}Vieille remarque de {#1}~: #2%
            %\end{tabular}\end{quote}%
}%
{}}

\newcommand{\highlightnotreviewed}{%
\color{blue}%
}%

\newcommand{\notreviewed}[2][]{%
\ifthenelse{\boolean{PrintRemarks}}{%
  \begin{changebar}%
  {\highlightnotreviewed #2}%
  \ifthenelse{\equal{#1}{}}{}{\footnote{#1}}%
  \end{changebar}%
}%
{}}

\ifthenelse{\boolean{PrintRemarks}}{%
\newenvironment{notreviewedenv}[1][]{%
  \begin{changebar}%
  \highlightnotreviewed%
  \ifthenelse{\equal{#1}{}}{}{\def\myrq{#1}}%
  \bgroup}%
 {\egroup%
  \ifthenelse{\isundefined{\myrq}}{}{\footnote{\myrq}}\end{changebar}}}%
{\excludecomment{notreviewedenv}}

%%% Commandes et environnements pour la version relative � l'implementation
\newcommand{\highlightnotimplemented}{%
\ifthenelse{\boolean{ColorImplementationRq}}{\color{red}}%
           {\ul}%
}%

\newcommand{\notimplemented}[2][]{%
\ifthenelse{\boolean{PrintImplementationRq}}{%
  \begin{changebar}%
  {\highlightnotimplemented #2}%
  \ifthenelse{\equal{#1}{}}{}{\footnote{#1}}%
  \end{changebar}%
}%
{#2}}

\newenvironment{notimplementedenv}[1][]{%
\ifthenelse{\boolean{PrintImplementationRq}}{%
  \begin{changebar}%
  \highlightnotimplemented%
  \ifthenelse{\equal{#1}{}}{}{\def\myrq{#1}}%
  \bgroup
}{}}%
{\ifthenelse{\boolean{PrintImplementationRq}}{%
    \egroup%
    \ifthenelse{\isundefined{\myrq}}{}{\footnote{\myrq}}\end{changebar}}{}}

%%% Environnements et commandes non conditionnelles
\newcommand{\experimental}{\textsc{Experimental}}

\newsavebox{\fmbox}
\newenvironment{cadre}
     {\begin{lrbox}{\fmbox}\begin{minipage}{0.98\textwidth}}
     {\end{minipage}\end{lrbox}\fbox{\usebox{\fmbox}}}


\newcommand{\keyword}[1]{\lstinline|#1|\xspace}
\newcommand{\keywordbs}[1]{\lstinline|\\#1|\xspace}

\newcommand{\integer}{\keyword{integer}}
\newcommand{\real}{\keyword{real}}
\newcommand{\bool}{\keyword{boolean}}

\newcommand{\assert}{\keyword{assert}}
\newcommand{\terminates}{\keyword{terminates}}
\newcommand{\assume}{\keyword{assume}}
\newcommand{\requires}{\keyword{requires}}
\newcommand{\ensures}{\keyword{ensures}}
\newcommand{\exits}{\keyword{exits}}
\newcommand{\returns}{\keyword{returns}}
\newcommand{\breaks}{\keyword{breaks}}
\newcommand{\continues}{\keyword{continues}}
\newcommand{\assumes}{\keyword{assumes}}
\newcommand{\assigns}{\keyword{assigns}}
\newcommand{\reads}{\keyword{reads}}
\newcommand{\decreases}{\keyword{decreases}}

\newcommand{\boundseparated}{\keywordbs{bound\_separated}}
\newcommand{\Exists}{\keywordbs{exists}~}
\newcommand{\Forall}{\keywordbs{forall}~}
\newcommand{\bslambda}{\keywordbs{lambda}~}
\newcommand{\freed}{\keywordbs{freed}}
\newcommand{\fresh}{\keywordbs{fresh}}
\newcommand{\fullseparated}{\keywordbs{full\_separated}}
\newcommand{\distinct}{\keywordbs{distinct}}
\newcommand{\Max}{\keyword{max}}
\newcommand{\nothing}{\keywordbs{nothing}}
\newcommand{\numof}{\keyword{num\_of}}
\newcommand{\offsetmin}{\keywordbs{offset\_min}}
\newcommand{\offsetmax}{\keywordbs{offset\_max}}
\newcommand{\old}{\keywordbs{old}}
\newcommand{\at}{\keywordbs{at}}

\newcommand{\If}{\keyword{if}~}
\newcommand{\Then}{~\keyword{then}~}
\newcommand{\Else}{~\keyword{else}~}
\newcommand{\For}{\keyword{for}~}
\newcommand{\While}{~\keyword{while}~}
\newcommand{\Do}{~\keyword{do}~}
\newcommand{\Let}{\keywordbs{let}~}
\newcommand{\Break}{\keyword{break}}
\newcommand{\Return}{\keyword{return}}
\newcommand{\Continue}{\keyword{continue}}

\newcommand{\exit}{\keyword{exit}}
\newcommand{\main}{\keyword{main}}
\newcommand{\void}{\keyword{void}}

\newcommand{\struct}{\keyword{struct}}
\newcommand{\union}{\keywordbs{union}}
\newcommand{\inter}{\keywordbs{inter}}
\newcommand{\typedef}{\keyword{typedef}}
\newcommand{\result}{\keywordbs{result}}
\newcommand{\separated}{\keywordbs{separated}}
\newcommand{\sizeof}{\keyword{sizeof}}
\newcommand{\strlen}{\keywordbs{strlen}}
\newcommand{\Sum}{\keyword{sum}}
\newcommand{\valid}{\keywordbs{valid}}
\newcommand{\validrange}{\keywordbs{valid\_range}}
\newcommand{\offset}{\keywordbs{offset}}
\newcommand{\blocklength}{\keywordbs{block\_length}}
\newcommand{\baseaddr}{\keywordbs{base\_addr}}
\newcommand{\comparable}{\keywordbs{comparable}}

\newcommand{\N}{\ensuremath{\mathbb{N}}}
\newcommand{\ra}{\ensuremath{\rightarrow}}
\newcommand{\la}{\ensuremath{\leftarrow}}

% BNF grammar
\newcommand{\indextt}[1]{\index{#1@\protect\keyword{#1}}}
\newcommand{\indexttbs}[1]{\index{#1@\protect\keywordbs{#1}}}
\newif\ifspace
\newif\ifnewentry
\newcommand{\addspace}{\ifspace ~ \spacefalse \fi}
\newcommand{\term}[2]{\addspace\hbox{\lstinline|#1|%
\ifthenelse{\equal{#2}{}}{}{\indexttbase{#2}{#1}}}\spacetrue}
\newcommand{\nonterm}[2]{%
  \ifthenelse{\equal{#2}{}}%
             {\addspace\hbox{\textsl{#1}\ifnewentry\index{grammar entries!\textsl{#1}}\fi}\spacetrue}%
             {\addspace\hbox{\textsl{#1}\footnote{#2}\ifnewentry\index{grammar entries!\textsl{#1}}\fi}\spacetrue}}
\newcommand{\repetstar}{$^*$\spacetrue}
\newcommand{\repetplus}{$^+$\spacetrue}
\newcommand{\repetone}{$^?$\spacetrue}
\newcommand{\lparen}{\addspace(}
\newcommand{\rparen}{)}
\newcommand{\orelse}{\addspace$\mid$\spacetrue}
\newcommand{\sep}{ \\[2mm] \spacefalse\newentrytrue}
\newcommand{\newl}{ \\ & & \spacefalse}
\newcommand{\alt}{ \\ & $\mid$ & \spacefalse}
\newcommand{\is}{ & $::=$ & \newentryfalse}
\newenvironment{syntax}{\begin{tabular}{@{}rrll@{}}\spacefalse\newentrytrue}{\end{tabular}}
\newcommand{\synt}[1]{$\spacefalse#1$}
\newcommand{\emptystring}{$\epsilon$}
\newcommand{\below}{See\; below}

% colors

\definecolor{darkgreen}{rgb}{0, 0.5, 0}

% theorems

\newtheorem{example}{Example}[chapter]

% for texttt

\newcommand{\bs}{\ensuremath{\backslash}}

\newcommand{\framacversion}{Carbon-20110201+dev
}

%Do not touch the following line. It is used in a Makefile hack to
%produce the ACSL documents for the ACSL working group.
%--\setboolean{PrintRemarks}{false}

%Do not touch the following line. It is used in a Makefile hack to
%produce the ACSL document shipped with the Frama-C distribution.
%--\setboolean{PrintImplementationRq}{false}

%\setboolean{ColorImplementationRq}{false}

%Already set in frama-c-book
%\usepackage[a4paper=true,pdftex,colorlinks=true,urlcolor=blue,pdfstartview=FitH]{hyperref}

%already set in frama-c-book
%\usepackage[T1]{fontenc}
%\usepackage{times}
\usepackage{amssymb}
\usepackage{graphicx}
%\usepackage{tikz}
\usepackage{color}
\usepackage{xspace}
\usepackage{makeidx}
\usepackage[normalem]{ulem}
\usepackage[leftbars]{changebar}
\usepackage{alltt}
\usepackage{comment}
\makeindex

%Already set in frama-c-book
%\setlength{\textheight}{240mm}
%\setlength{\topmargin}{-10mm}
%\setlength{\textwidth}{160mm}
%\setlength{\oddsidemargin}{0mm}
%\setlength{\evensidemargin}{0mm}
\newcommand{\version}{1.4}

\renewcommand{\textfraction}{0.01}
\renewcommand{\topfraction}{0.99}
\renewcommand{\bottomfraction}{0.99}

% already taken care of in frama-c-book
% \usepackage{fancyhdr}
% \pagestyle{fancyplain}
% \renewcommand{\footrulewidth}{0.4pt}
% \addtolength{\headheight}{2pt}
% \addtolength{\headwidth}{1cm}
% \renewcommand{\chaptermark}[1]{\markboth{#1}{}}
% \renewcommand{\sectionmark}[1]{\markright{\thesection\ #1}}
% \lhead[\fancyplain{}{\bfseries\thepage}]{\fancyplain{}{\bfseries\rightmark}}
% \chead{}
% \rhead[\fancyplain{}{\bfseries\leftmark}]{\fancyplain{}{\bfseries\thepage}}
% \lfoot{\fancyplain{}{ANSI/ISO C Specification Language}}
% \cfoot{}
% \rfoot{\fancyplain{}{CAT RNTL project}}

\begin{document}
\sloppy
\hbadness=10000

\ifthenelse{\boolean{PrintImplementationRq}}%
  {\coverpage{ACSL Implementation in \framacversion{}}}%
  {\coverpage{ACSL: ANSI/ISO C Specification Language}}

\begin{titlepage}
\includegraphics[height=14mm]{cealistlogo.jpg}
\hfill
\includegraphics[height=14mm]{inriasaclaylogo.png}
\vfill
\title{ACSL: ANSI/ISO C Specification Language}%
{Version \version{}\ifthenelse{\boolean{PrintImplementationRq}}%
{~--~\framacversion}{}}
\author{Patrick Baudin$^1$, Pascal Cuoq$^1$, Jean-Christophe Filli\^atre$^{4,3}$, Claude March\'e$^{3,4}$,\\ Benjamin Monate$^1$, Yannick Moy$^{2,4,3}$, Virgile Prevosto$^1$}

\begin{tabular}{l}
$^1$ CEA LIST, Software Reliability Laboratory, Saclay, F-91191 \\
$^2$ France T\'el\'ecom, Lannion, F-22307 \\
$^3$ INRIA Saclay - \^Ile-de-France, ProVal, Orsay, F-91893 \\
$^4$ LRI, Univ Paris-Sud, CNRS, Orsay, F-91405
\end{tabular}
\vfill
\begin{flushleft}
  \textcopyright 2009 CEA LIST and INRIA

  This work has been supported by the `CAT' ANR project
  (ANR-05-RNTL-0030x) and by the ANR CIFRE contract 2005/973.
\end{flushleft}
\end{titlepage}

%%Contents should open on right
\cleardoublepage
\label{chap:contents}
\tableofcontents

\chapter*{Foreword}

% This is a preliminary design of the ACSL language, a deliverable of
% the task 7.2 of the ANR RNTL project CAT
% (\url{http://www.rntl.org/projet/resume2005/cat.htm}). In this
% project, a reference implementation of ACSL is provided: the Frama-C
% platform (\url{http://frama-c.cea.fr}).

This is the version \version{} of ACSL design. Several features may still
evolve in the future. In particular, some features in this document
are considered \emph{experimental}, meaning that their syntax and
semantics is not already fixed.  These features are marked with
\experimental.  They must also be considered as advanced features,
which are not supposed to be useful for a basic use of that
specification language.

\section*{Acknowledgements}

We gratefully thank all the people who contributed to this document:
Sylvie Boldo,
Jean-Louis Cola\c{c}o,
Pierre Cr\'egut,
David Delmas,
St\'ephane Duprat,
Arnaud Gotlieb,
Thierry Hubert,
Dillon Pariente,
Pierre Rousseau,
Julien Signoles,
Jean Souyris.

%; whizzy-master "main.tex"
\chapter{Introduction}

This document is a reference manual for
\ifthenelse{\boolean{PrintImplementationRq}}%
{the E-ACSL implementation provided by the Frama-C
  framework~\cite{frama-c}.}%
{E-ACSL.}
E-ACSL is an acronym for ``Executable ANSI/ISO C
Specification Language''. It is an ``executable'' subset of
ACSL~\cite{acsl} implemented in the \framac platform~\cite{framac}. Contrary to
ACSL, each E-ACSL specification is executable: it may be evaluated at runtime.

In this document, we assume that the reader has a good knowledge of both
ACSL~\cite{acsl} and the ANSI C programming language~\cite{KR88,standardc99}.

\section{Organization of this document}

This document is organized in the very same way that the reference manual of
ACSL~\cite{acsl}.

Instead of being a fully new reference manual, this document points out the
differences between E-ACSL and ACSL. Each E-ACSL construct which is not pointed
out must be considered to have the very same semantics than its ACSL
counterpart. For clarity, each relevant grammar rules are given in BNF form
in separate figures like the \acsl reference manual does.

\section{Generalities about Annotations}\label{sec:gener-about-annot}
\nodiff

\section{Notations for grammars}
\nodiff


%; whizzy-master "main.tex"

\chapter{Specification language}
\label{chap:base}

\section{Lexical rules}

\section{Logic expressions}
\label{sec:expressions}

\subsection{Operators precedence}

\subsection{Semantics}
\label{sec:twovaluedlogic}

\subsection{Integer arithmetic and machine integers}

\subsubsection{Hexadecimal and octal constants}

\subsubsection{Casts and  overflows}\index{cast}

\subsubsection{Quantification on C integral types}
\label{sec:quantification}

\subsubsection{Size of C integer types}

\subsubsection{Enum types}

\subsubsection{Bitwise operations}

\subsection{Real numbers and floating point numbers}

\subsubsection{Casts, infinity and NaNs}

\subsubsection{Quantification}


\subsubsection{Mathematical functions}

\subsubsection{Exact computations}

\subsection{C arrays and pointers}

\subsubsection{Address operator, array access, pointer arithmetic and dereferencing}
\label{sec:address}

\subsubsection{Function pointers}

\subsubsection{Functional updates}


\subsubsection{C aggregate types}


\subsection{String literals}


\section{Function contracts}
\label{sec:fn-behavior}
\index{function contract}\index{contract}


\subsection{Built-in constructs %
  \texorpdfstring{\old}{\textbackslash{}old} %
 and \texorpdfstring{\result}{\textbackslash{}result}}


\subsection{Simple function contracts}
\label{sec:simplecontracts}

\subsection{Contracts with named behaviors}
\label{subsec:behaviors}
\index{function behavior}\index{behavior}

\subsubsection{Completeness of behaviors}
\label{sec:compl-behav}

\subsection{Memory locations and sets of terms}
\label{sec:locations}

\subsection{Default contracts, multiple contracts}
\label{sec:multiplecontracts}

\section{Statement annotations}
\index{annotation}

\subsection{Assertions}
\indextt{assert}

\subsection{Loop annotations}
\label{sec:loop_annot}

\subsubsection{Loop invariants}
\index{invariant}

\subsubsection{Loop variants}
\index{variant}\index{termination}

\subsubsection{General inductive invariants}
\index{invariant}

\subsection{Built-in construct \texorpdfstring{\at}{\textbackslash{}at}}
\label{sec:at}

\subsubsection*{Default logic labels}\label{sec:default-logic-labels}

\subsection{Statement contracts}
\label{sec:statement_contract}
\index{statement contract}\index{contract}

\section{Termination}
\label{sec:termination}
\index{termination}

\subsection{Integer measures}
\label{sec:integermeasures}
\indexttbs{decreases}\indexttbs{variant}

\subsection{General measures}
\label{sec:generalmeasures}

\subsection{Recursive function calls}

\subsection{Non-terminating functions}
\label{sec:non-term-funct}
\indextt{terminates}
\experimental

\section{Logic specifications}
\label{sec:logicspec}
\index{logic specification}\index{specification}

\subsection{Predicate and function definitions}

\subsection{Lemmas}

\subsection{Inductive predicates}
\label{sec:inductivepredicates}
\index{inductive predicates}


\subsection{Axiomatic definitions}

\subsection{Polymorphic logic types}\label{sec:polym-logic-types}
\index{type!polymorphic}
\index{polymorphism}
\experimental

\subsection{Recursive logic definitions}
\index{recursion}

\subsection{Higher-order logic constructions}
\label{sec:higherorder}

\experimental

\subsection{Concrete logic types}\label{sec:concrete-logic-types}
\index{type!concrete}
\experimental

\subsection{Hybrid functions and predicates}
\label{sec:logicalstates}
\index{hybrid!function}
\index{hybrid!predicate}

\subsection{Memory footprint specification: \texorpdfstring{\lstinline|reads|}{reads} clause}

\experimental

\subsection{Specification Modules}
\label{sec:specmodules}
\index{module}

\section{Pointers and physical adressing}
\label{sec:pointers}

\subsection{Memory blocks and pointer dereferencing}
\label{subsec:memory}

\subsection{Separation}
\label{sec:separated}

\experimental

\subsection{Allocation and deallocation}

\experimental

\section{Sets as first-class values}
\index{location}

\section{Abrupt termination}
\label{sec:abrupt-clauses}
\index{abrupt clause}

\experimental

\section{Dependencies information}
\label{sec:func-dep}

\experimental

\section{Data invariants}
\label{sec:invariants}
\index{data invariant}\index{global invariant}\index{type invariant}
\index{invariant!data}\index{invariant!global}\index{invariant!type}

\subsection{Semantics}

\subsection{Model variables and model fields}
\index{model}

\section{Ghost variables and statements}
\label{sec:ghost}
\index{ghost}

\subsection{Volatile variables}\label{sec:volatile-variables}
\index{volatile}
\experimental


\section{Undefined values, dangling pointers}

\subsection{Initialization}
\indexttbs{initialized}

\subsection{Unspecified values}
\indexttbs{specified}


\chapter{Libraries}
\label{chap:lib}

\emph{Disclaimer:} this chapter is yet empty. It is left here to give an idea of
what the final document will look and to be consistent with the \acsl reference
manual~\cite{acsl}.



\chapter{Conclusion}

This document presents a Behavioral Interface Specification Language
for ANSI \lang{} source code. It provides a common basis that can be
shared among different tools.
The specification language described here is intended to evolve in the
future and remain open to additional constructions.
One interesting possible extension regards ``temporal''
properties in a large sense, such as liveness properties, which can
sometimes be simulated by regular specifications with ghost
variables~\cite{giorgetti06fase}, or properties on evolution of data
over the time, such as the history constraints of JML, or in the Lustre
assertion language.

%%% Local Variables:
%%% mode: latex
%%% TeX-PDF-mode: t
%%% TeX-master: "main"
%%% End:


\appendix

\chapter{Appendices}
\label{chap:appendix}

%\section{Quick reference card}

%\todo{by Benjamin}

\section{Glossary}
\label{sec:glossary}

\begin{description}
\item[pure expressions] \index{pure expression} In ACSL setting, a
  \emph{pure} expression is a C expression which contains no assignments, no
  incrementation operator \lstinline|++| or \lstinline|--|, no function call,
  and no access to a volatile object. The set of pure expression is a
  subset of the set of C expressions without side effect (C
  standard~\cite{KR88,standardc99}, \S 5.1.2.3, alinea 2).

\item[left-values] \index{left-value}\index{lvalue}

  A \emph{left-value} (\emph{lvalue} for short) is an expression which
  denotes some place in the memory during program execution, either on
  the stack, on the heap, or in the static data segment. It can be
  either a variable identifier or an expression of the form \lstinline|*e|,
  \lstinline|e[e]|, \lstinline|e.id| or \lstinline|e->id|, where
  \lstinline |e| is any expression and \lstinline|id| a field name.
  See C standard, \S 6.3.2.1 for a more
  detailed description of lvalues.

  A \emph{modifiable lvalue} is an lvalue allowed in the left part of
  an assignment. In essence, all lvalues are modifiable except
  variables declared as \texttt{const} or of some array type with
  explicit length.

  \oldremark{DP}{
De'tailler (avec un exemple), meme si c'est dans la
    norme, la difference entre lvalue modifiable et non-modifiable
  }


\item[pre-state and post-state]
    \index{pre-state}\index{post-state}

    For a given function call, the \emph{pre-state} denotes the
    program state at the beginning of the call, including the
    current values for the function parameters. The \emph{post-state}
    denotes the program state at the return of the call.

\item[function behavior] \index{function behavior} \index{behavior}

  A \emph{function behavior} (\emph{behavior} for short) is a set of
  properties relating the pre-state and the post-state for a
  possibly restricted set of pre-states (behavior \emph{assumptions}).

\item[function contract] \index{function contract} \index{contract}

  A \emph{function contract} (\emph{contract} for short) forms a
  specification of a function, consisting of the combination of a
  precondition (a requirement on the pre-state for any caller to that
  function), a collection of behaviors, and possibly a measure in case
  of a recursive function.

\end{description}

\section{Comparison with JML}
\label{sec:comp-jml}

Although we took our inspiration from the Java Modeling Language (aka
JML~\cite{jmlhomepage}), ACSL is notably different from JML in
two crucial aspects:

\begin{itemize}
\item ACSL is a BISL for C, a low-level structured language, while JML
  is a BISL for Java, an object-oriented inheritance-based high-level
  language. Not only are the language features not the same between Java and C, but the
  programming styles and idioms are very different, which then entails
  different ways of specifying behaviors. In particular, C has no
  inheritance or exceptions, and no language support for the simplest
  properties on memory (\emph{e.g.}, the size of an allocated memory block).
\item JML also supports runtime assertion checking (RAC) when typing,
  static analysis and automatic deductive verification fail. The
  example of CCured~\cite{necula02ccured,condit03ccured}, which also adds
  strong typing to C by relying on RAC, shows that it is not possible
  to do it in a modular way. Indeed, it is necessary to modify the
  layout of C data structures for RAC, which is not modular. The
  follow-up project Deputy~\cite{condit07deputy} thus reduces the
  checking power of annotations in order to preserve modularity.  In contrast, we choose not to restrain the power of annotations
  (\emph{e.g.}, all first order logic formulas are allowed). To that end, we
  rely on manual deductive verification using an interactive theorem
  prover (\emph{e.g.}, Coq) when every other technique fails.
\end{itemize}

\noindent
In the remainder of this chapter, we describe these differences in
further detail.

\subsection{Low-level language vs. inheritance-based one}

\subsubsection*{No inherited specifications}

JML has a core notion of specification inheritance, which
enables support for behavioral subtyping, by applying
 specifications of parent methods to overriding methods.  Inheritance
combined with visibility and modularity account for a number of
complex features in JML (\emph{e.g.}, \verb|spec_public| modifier, data
groups, represents clauses, etc), that are necessary to express the
desired inheritance-related specifications while respecting visibility
and modularity. Since C has no inheritance, these intricacies are
avoided in ACSL.

\subsubsection*{Error handling without exceptions}
\label{sec:errorhandling}

The usual way of signaling errors in Java is through
exceptions. Therefore, JML specifications are tailored to express
exceptional postconditions, depending on the exception raised. Since C
has no exceptions, ACSL does not use exceptional
specifications. Instead, C programmers typically signal errors by
returning special values, as is mandated in various ways by the C standard.

\begin{example}
In \S 7.12.1 of the standard, it is said that functions in <math.h>
signal errors as follows:
``On a domain error, [...] the integer expression errno
acquires the value EDOM.''
\end{example}

\begin{example}
In \S 7.19.5.1 of the standard, it is said that function fclose signals
errors as follows:
``The fclose function returns [...] EOF if any errors were detected.''
\end{example}

\begin{example}
In \S 7.19.6.1 of the standard, it is said that function fprintf
signals errors as follows:
``The fprintf function returns [...] a negative value if an output or
encoding error occurred.''
\end{example}

\begin{example}
In \S 7.20.3 of the standard, it is said that memory management functions
signal errors as follows:
``If the space cannot be allocated, a null pointer is returned.''
\end{example}

As shown by these few examples, there is no unique way to signal
errors in the C standard library, not to mention errors from user-defined
functions. But since errors are signaled by returning special values, it
is sufficient to write an appropriate postcondition:

\begin{listing-nonumber}
/*@ ensures \result == error_value || normal_postcondition; */
\end{listing-nonumber}

% \noindent
% A tool could easily set error conditions aside, by providing an
% appropriate extension of behaviors, \emph{e.g}, using a new keyword
% \verb|failswith|:

% \begin{flushleft}\ttfamily
% /*@ failswith $\mathit{\result == error\_value}$; \\
% ~~@ ensures $\mathit{normal\_postcondition}$; \\
% ~~@*/
% \end{flushleft}


%\input{fwrite-malloc.pp}


\subsubsection*{C contracts are not Java ones}

In Java, the precondition of the following function that nullifies an
array of characters is always true. Even if there was a precondition
on the length of array {\ttfamily a}, it could easily be expressed using
the Java expression {\ttfamily a.length} that gives the dynamic length
of array {\ttfamily a}.

\begin{listing}{1}
public static void Java_nullify(char[] a) {
  if (a == null) return;
  for (int i = 0; i < a.length; ++i) {
    a[i] = 0;
  }
}
\end{listing}

On the contrary, the precondition of the same function in C, whose
definition follows, is more involved. First, remark that the
C programmer has to add an extra argument for the size of the array,
or rather a lower bound on this array size.

\begin{listing}{1}
void C_nullify(char* a, unsigned int n) {
  int i;
  if (n == 0) return;
  for (i = 0; i < n; ++i) {
    a[i] = 0;
  }
}
\end{listing}

\noindent
A correct precondition for this function is the following:

\begin{listing-nonumber}
/*@ requires \valid(a + 0..(n-1)); */
\end{listing-nonumber}

where predicate \valid is the one defined in Section~\ref{subsec:memory}.
(note that \lstinline|\valid(a + 0..(-1))| is the same as
\lstinline|\valid(\empty)| and thus is true regardless of the validity of
\lstinline|a| itself).
When \lstinline|n| is 0, \lstinline|a| does
not need to be valid at all, and when \lstinline|n| is strictly
positive, \lstinline|a| must point to an array of size at least
\lstinline|n|. To make it more obvious, the C programmer adopted a
defensive programming style, which returns immediately when \lstinline|n| is
0. We can duplicate this in the specification:

\begin{listing-nonumber}
/*@ requires n == 0 || \valid(a + 0..(n-1)); */
\end{listing-nonumber}

Many memory requirements are only necessary for some paths
through the function, which correspond to some particular
behaviors, selected according to some tests performed along the
corresponding paths. Since C has no memory
primitives, these tests involve other variables that the C programmer
adds to track additional information, such as {\ttfamily n} in our example.

To make it easier, it is possible in ACSL to distinguish between the
\lstinline|assumes| part of a behavior, that specifies the tests that need
to succeed for this behavior to apply, and the \lstinline|requires| part
that specifies the additional assumptions that must be true when a
behavior applies. The specification for our example can then be
translated into:

\begin{listing}{1}
/*@ behavior n_is_null:
  @   assumes n == 0;
  @ behavior n_is_not_null:
  @   assumes n > 0;
  @   requires \valid(a + 0..(n-1));
  @*/
\end{listing}

This is equivalent to the previous requirement, except here behaviors
can be completed with postconditions that belong to one behavior only.

\subsubsection*{ACSL contracts vs. JML ones}

In JML, the set of stated behaviors is assumed to cover all
permitted uses of the function; any calling context in which none of the requires preconditions are true would be identified as an error.
In ACSL, the set of behaviors for a function do not
necessarily cover all cases of use for this function, as mentioned in
Section~\ref{subsec:behaviors}. This allows for partial
specifications. In the example above, our two behaviors are clearly mutually exclusive,
and, since \lstinline|n| is an \lstinline|unsigned int|, 
they cover all the possible cases. We could have specified that as well, by
adding the following lines in the contract (see
Section~\ref{sec:compl-behav}).
\begin{listing}{last}
  @ ...
  @ disjoint behaviors;
  @ complete behaviors;
  @*/
\end{listing}

To fully understand the difference between specifications in ACSL and
JML, we detail below the requirements on the pre-state and
the guarantees in the post-state given by behaviors in JML and ACSL.

A JML contract is either \emph{lightweight} or \emph{heavyweight}.
For the purpose of our comparison, it is sufficient to know that a
lightweight contract is syntactic sugar for a single specific 
heavyweight contract; a contract can have multiple heavyweight behaviors and these can be nested.
Here is a hypothetical JML contract:
\begin{listing}{1}
/*@ behavior $x_1$:
  @   requires $A_1$;
  @   requires $R_1$;
  @   ensures $E_1$;
  @ behavior $x_2$:
  @   requires $A_2$;
  @   requires $R_2$;
  @   ensures $E_2$;
  @*/
\end{listing}
It assumes that the pre-state satisfies the condition:
\begin{listing-nonumber}
(($A_1$ && $R_1$) || ($A_2$ && $R_2$))
\end{listing-nonumber}
and guarantees that the following condition holds in post-state:
\begin{listing-nonumber}
  (\old($A_1$ && $R_1$) ==> $E_1$) && (\old($A_2$ && $R_2$) ==> $E_2$)
\end{listing-nonumber}
Note particularly that the pre-state is required to satisfy
the precondition of at least one behavior.

Here is now a syntactically similar ACSL specification:

\begin{listing}{1}
/*@ requires $P_1$;
  @ requires $P_2$;
  @ ensures  $Q_1$;
  @ ensures  $Q_2$;
  @ behavior $x_1$:
  @   assumes $A_1$;
  @   requires $R_1$;
  @   ensures $E_1$;
  @ behavior $x_2$:
  @   assumes $A_2$;
  @   requires $R_2$;
  @   ensures $E_2$;
  @*/
\end{listing}

\noindent
Syntactically, the only difference with the JML specification is the
addition of the \lstinline|assumes| clauses and allowing an 
anonymous behavior at the beginning of the contract. Rewriting the anonymous behavior with a name gives

\begin{listing}{1}
	/*@ 
	@ behavior $x_0$:
	@   assumes \true;
	@   requires $P_1$;
	@   requires $P_2$;
	@   ensures  $Q_1$;
	@   ensures  $Q_2$;
	@ behavior $x_1$:
	@   assumes $A_1$;
	@   requires $R_1$;
	@   ensures $E_1$;
	@ behavior $x_2$:
	@   assumes $A_2$;
	@   requires $R_2$;
	@   ensures $E_2$;
	@*/
\end{listing}

\noindent
Its translation to assume-guarantee is however quite different than JML.
It assumes the pre-state satisfies the condition

\begin{listing-nonumber}
  (\true ==> ($P_1$ && $P_2$)) && ($A_1$ ==> $R_1$) && ($A_2$ ==> $R_2$)
\end{listing-nonumber}
Here, it is acceptable that none of the behaviors are active (that is, that none of the \lstinline|assumes| clauses are true, even without the unnamed behavior). In that case there is no post-condition guarantee either.

The contract guarantees that the following condition holds in the post-state:

\begin{listing-nonumber}
(\true ==> ($Q_1$ && $Q_2$)) && (\old($A_1$) ==> $E_1$) && (\old($A_2$) ==> $E_2$)
\end{listing-nonumber}

Thus, ACSL allows distinguishing between the clauses that control
which behavior is active (the \lstinline|assumes| clauses) and the
clauses that are preconditions for a particular behavior (the internal
\lstinline|requires| clauses). 

In addition, as mentioned above, there is
by default no requirement in ACSL for the specification to be complete. In JML an incomplete specification may cause a warning in a calling context; partial behavior is specified by an explicitly underspecified postcondition. In ACSL, an incomplete specification specifies partial behavior; a warning for a particular behavior is produced by a \lstinline|requires \false;| clause.

\subsection{Deductive verification vs. RAC}

\subsubsection*{Sugar-free behaviors}

As explained in detail in~\cite{raghavan00desugaring}, JML
heavyweight behaviors can be viewed as syntactic sugar that can be translated automatically into more basic
contracts consisting mostly of pre- and postconditions and frame
conditions.  This allows complex nesting of behaviors from the user
point of view, while tools only have to deal with basic contracts. In
particular, older tools on JML used this desugaring process, such as
the Common JML tools to do assertion checking, unit testing,
etc. (see~\cite{leavens00jml}) and the tool ESC/Java2 for
automatic deductive verification of JML specifications
(see~\cite{Kiniry-Cok05}).

One issue with such a desugaring approach is the complexity of the
transformations involved, as \emph{e.g.} for desugaring assignable clauses
between multiple \textit{spec-cases} in
JML~\cite{raghavan00desugaring}.  Another issue is precisely that
tools only see one global contract, instead of multiple independent
behaviors, that could be analyzed separately in more detail.
Instead, we favor the view that a function implements multiple
behaviors, that can be analyzed separately if a tool feels like
it. Therefore, we do not intend to provide a desugaring process.
Indeed, the current JML tool, OpenJML ~\cite{Cok-2011-OpenJML,Cok-2014-OpenJML}, also does only a partial desugaring, which at minimum is able to give more informative error messages when proof attempts fail.

\subsubsection*{Axiomatized functions in specifications}

JML allows pure Java methods to be called in
specifications~\cite{leavens00preliminary}. This avoids having
to write essentially duplicate logical functions that mimic Java functions. It is also useful when relying on RAC: methods called should be defined
so that the runtime can call them, and they should not have
side-effects in order not to pollute the program they are supposed to
annotate. 
JML also permits model (logical) functions to be
used in specifications; if the model function does not have
a body, then RAC cannot be used. But for deductive verification, 
the properties of a model function can be specified axiomatically.

ACSL focuses on deductive verification and currently only allows calls to logical functions in
specifications. These functions may be defined, like program functions, but
they may also be only declared (with a suitable declaration of \reads
clause) and their behavior defined through an axiomatization.
This makes for richer specifications that may be useful either in
automatic or in manual deductive verification.


\subsection{Syntactic differences}

The following table summarizes the difference between JML and ACSL keywords, when the intent is the same, although minor differences
might exist.
\begin{center}
\begin{tabular}{|l|l|}
\hline
  JML                  & ACSL \\ \hline
  modifiable, assignable           & assigns \\
  measured\_by         & decreases \\
  loop\_invariant      & loop invariant \\
  decreases            & loop variant \\
  \lstinline|(\forall $\tau$ x ; P ; Q)| &
       \lstinline|(\forall $\tau$ x ; P ==> Q)| \\
  \lstinline|(\exists $\tau$ x ; P ; Q)| &
        \lstinline|(\exists $\tau$ x ; P && Q)| \\
  \lstinline|\max $\tau$ x ; a <= x <= b ; f)| &
        \lstinline|\max(a,b,\lambda $\tau$ x ; f)| \\
  \hline
\end{tabular}
\end{center}

%%% Local Variables:
%%% mode: latex
%%% TeX-PDF-mode: t
%%% TeX-master: "main"
%%% End:


\section{Typing rules}
\label{sec:typingrules}

Disclaimer: this section is unfinished, it is left here just to give an idea of what it will look like when completed.

\subsection{Rules for terms}

Integer promotion:
\[
\frac{\Gamma \vdash e : \tau}{\Gamma \vdash e : \integer}
\]
if $\tau$ is any C integer type \verb|char|, \verb|short|, \verb|int|, or \verb|long|, whatever attribute they have, in particular signed or unsigned

Variables:
\[
\frac{}{\Gamma \vdash id : \tau} \mbox{ if $id:\tau\in\Gamma$}
\]

Unary integer operations:
\[
\frac{\Gamma \vdash t : \integer}{\Gamma \vdash op~t : \integer} \mbox{ if $op\in \{+,-,\sim\}$}
\]

Boolean negation:
\[
\frac{\Gamma \vdash t : \bool}{\Gamma \vdash !~t : \bool}
\]

Pointer dereferencing:
\[
\frac{\Gamma \vdash t : \tau*}{\Gamma \vdash *t : \tau}
\]

Address operator:
\[
\frac{\Gamma \vdash t : \tau}{\Gamma \vdash \&t : \tau*}
\]

Binary
\[
\frac{\Gamma \vdash t_1 : \integer\qquad\Gamma \vdash t_2 : \integer}{\Gamma \vdash t_1~op~t_2 : \integer} \mbox{ if $op\in \{+,-,*,/,\%\}$}
\]
\[
\frac{\Gamma \vdash t_1 : \real\qquad\Gamma \vdash t_2 : \real}{\Gamma \vdash t_1~op~t_2 : \real} \mbox{ if $op\in \{+,-,*,/\}$}
\]
\[
\frac{\Gamma \vdash t_1 : \integer\qquad\Gamma \vdash t_2 : \integer}{\Gamma \vdash t_1~op~t_2 : \bool} \mbox{ if $op\in \{==,!=,<=,<,>=,>\}$}
\]
\[
\frac{\Gamma \vdash t_1 : \real\qquad\Gamma \vdash t_2 : \real}{\Gamma \vdash t_1~op~t_2 : \bool} \mbox{ if $op\in \{==,!=,<=,<,>=,>\}$}
\]
\[
\frac{\Gamma \vdash t_1 : \tau*\qquad\Gamma \vdash t_2 : \tau*}{\Gamma \vdash t_1~op~t_2 : \bool} \mbox{ if $op\in \{==,!=,<=,<,>=,>\}$}
\]

(to be continued)


\subsection{Typing rules for sets}
%VP: A-t-on besoin d'un logic environment?
We consider the typing judgement $\Gamma,\Lambda \vdash s : \tau,b$
meaning that $s$ is a set of terms of type $\tau$, which is moreover a
set of locations if the boolean $b$ is true.
$\Gamma$ is the C environment and $\Lambda$ is the logic environment.

Rules:
\[
\frac{}{\Gamma,\Lambda \vdash id : \tau,true} \mbox{ if $id:\tau \in \Gamma$}
\]
\[
\frac{}{\Gamma,\Lambda \vdash id : \tau,true} \mbox{ if $id:\tau \in \Lambda$}
\]
\[
\frac{\Gamma,\Lambda\vdash s:\tau*,b}{\Gamma,\Lambda \vdash *s: \tau,true}
\]
%VP: cette regle ne veut rien dire.
%CM maintenant si, si l'environnement donne les types des champs
\[
\frac{id:\tau \quad s: set<struct~S*>}{\vdash s->id : set<\tau>}
\]
\[
\frac{\Gamma,b\cup \Lambda \vdash e: tset \tau}
{\Gamma,\Lambda\vdash \{ e \mid b ; P \} : tset \tau }
\]
\[
\frac{\Gamma,\Lambda\vdash e_1:\tau,b \quad \Gamma,\Lambda\vdash e_2:\tau,b}
{\Gamma,\Lambda\vdash e_1,e_2: \tau,b}
\]

%%% Local Variables:
%%% mode: latex
%%% TeX-PDF-mode: t
%%% TeX-master: "main"
%%% End:


\section{Specification Templates}\label{sec:spec-templ}
This section describes some common issues that may occur when writing
an ACSL specification and proposes some solution to overcome them

\subsection{Accessing a C variable that is masked}

The situation may happen where it is necessary to refer
in an annotation to a C variable that is masked at that point.
For instance,
a function contract may need to refer to a global variable that has the
same name as a function parameter, as in the following code:

\listinginput{1}{glob_var_masked.c}

In order to write the \verb|assigns| clause for \verb|f|, we must
access the global variable \verb|x|, since \verb|f| calls \verb|g|,
which can modify \verb|x|. This is not possible with C scoping rules,
as \verb|x| refers to the parameter of \verb|f| in the scope of the
function.

A solution is to use a ghost pointer to \verb|x|, as shown in
the following code:

\listinginput{1}{glob_var_masked_sol.c}


%%% Local Variables:
%%% mode: latex
%%% TeX-PDF-mode: t
%%% TeX-master: "main"
%%% End:


\section{Illustrative example}

This is an attempt to define an example for ACSL, much as the Purse
example in JML description papers.  It is a memory allocator, whose
main functions are \lstinline|memory_alloc| and \lstinline|memory_free|, to
respectively allocate and deallocate memory.  The goal is to exercise
as much as possible of ACSL.

%\input{acsl_allocator.pp}
\listinginput{1}{acsl_allocator.c}

\section{Changes}

\subsection{Version 1.x}
\begin{itemize}
\item Fixed typos in the examples corresponding to features
  implemented in Frama-C
\end{itemize}

\subsection{Version 1.4}
\begin{itemize}
\item Introduction of \lstinline|axiomatic| to gather predicates, logic
  functions, and their defining axioms.
\item added specification templates appendix for common specification issues
\item use of sets as first-class term as been precised
\item fixed semantics of predicated \lstinline|\separated|
\end{itemize}

\subsection{Version 1.3}
\begin{itemize}
\item Functional update of structures
\item Terminates clause in function behaviors
\item Typos reported by David Mentre

\end{itemize}

\subsection{Version 1.2}
This is the first public release of this document.

\cleardoublepage
\addcontentsline{toc}{chapter}{\bibname}
\bibliographystyle{plain}
%\IfFileExists{biblio-demons.tex}{%
%\input{biblio-demons.tex}
%}{
\bibliography{./biblio}
%}

\cleardoublepage
\addcontentsline{toc}{chapter}{\listfigurename}
\listoffigures

\cleardoublepage
\addcontentsline{toc}{chapter}{\indexname}
\printindex


\end{document}

%%% Local Variables:
%%% mode: latex
%%% TeX-PDF-mode: t
%%% TeX-master: t
%%% End:
