\documentclass[a4paper,11pt,twoside,openright]{report}

\usepackage[T1]{fontenc}
\usepackage{times}
\usepackage{url,amssymb}
\usepackage{graphicx}
\usepackage{color}
\usepackage{xspace}
\usepackage{makeidx}
\makeindex

\newcommand{\experimental}{\textsc{Experimental}}

\newcommand{\highlightnotimplemented}[1]{
\ifthenelse{\boolean{ColorImplementationRq}}{\textcolor{red}{#1}}%
           {\uwave{#1}}
}

\newcommand{\notimplemented}[2][]{%
\ifthenelse{\boolean{PrintImplementationRq}}{%
  \begin{changebar}%
  \highlightnotimplemented{#2}%
  \ifthenelse{\equal{#1}{}}{}{\footnote{#1}}%
  \end{changebar}%
}%
{#2}}


\newsavebox{\fmbox}
\newenvironment{cadre}
     {\begin{lrbox}{\fmbox}\begin{minipage}{0.98\textwidth}}
     {\end{minipage}\end{lrbox}\fbox{\usebox{\fmbox}}}

% \newenvironment{todo}{\begin{quote}
%     \begin{tabular}{||p{0.8\textwidth}}
% TODO~:\itshape}{\end{tabular}\end{quote}}

%\newenvironment{remark}[1]{\begin{quote}\itshape
%    \begin{tabular}{||p{0.8\textwidth}}
%Remarque de {#1}~:}{\end{tabular}\end{quote}}

\newcommand{\oldremark}[2]{%
%\begin{quote}\itshape
%    \begin{tabular}{||p{0.8\textwidth}}
%Vieille remarque de {#1}~: #2
%\end{tabular}\end{quote}
}

\newcommand{\keyword}[1]{\texttt{#1}\xspace}

\newcommand{\integer}{\keyword{integer}}
\newcommand{\real}{\keyword{real}}
\newcommand{\bool}{\keyword{boolean}}
\newcommand{\assert}{\keyword{assert}}
\newcommand{\assume}{\keyword{assume}}
\newcommand{\requires}{\keyword{requires}}
\newcommand{\ensures}{\keyword{ensures}}
\newcommand{\assumes}{\keyword{assumes}}
\newcommand{\assigns}{\keyword{assigns}}
\newcommand{\reads}{\keyword{reads}}
\newcommand{\decreases}{\keyword{decreases}}
\newcommand{\boundseparated}{\keyword{{\textbackslash}bound\_separated}}
\newcommand{\Exists}{\keyword{{\textbackslash}exists}~}
\newcommand{\Forall}{\keyword{{\textbackslash}forall}~}
\renewcommand{\Lambda}{\keyword{{\textbackslash}lambda}~}
\newcommand{\freed}{\keyword{{\textbackslash}freed}}
\newcommand{\fresh}{\keyword{{\textbackslash}fresh}}
\newcommand{\fullseparated}{\keyword{{\textbackslash}full\_separated}}
\newcommand{\distinct}{\keyword{{\textbackslash}distinct}}
\newcommand{\Max}{\keyword{max}}
\newcommand{\nothing}{\keyword{{\textbackslash}nothing}}
\newcommand{\numof}{\keyword{num\_of}}
\newcommand{\offsetmin}{\keyword{{\textbackslash}offset\_min}}
\newcommand{\offsetmax}{\keyword{{\textbackslash}offset\_max}}
\newcommand{\old}{\keyword{{\textbackslash}old}}
\newcommand{\at}{\keyword{{\textbackslash}at}}

\newcommand{\If}{\keyword{if}~}
\newcommand{\Then}{~\keyword{then}~}
\newcommand{\Else}{~\keyword{else}~}
\newcommand{\For}{\keyword{for}~}
\newcommand{\While}{~\keyword{while}~}
\newcommand{\Do}{~\keyword{do}~}
\newcommand{\Let}{\keyword{\textbackslash{}let}~}

\newcommand{\struct}{\keyword{struct}}
\newcommand{\union}{\keyword{union}}
\newcommand{\inter}{\keyword{inter}}
\newcommand{\typedef}{\keyword{typedef}}
\newcommand{\result}{\keyword{{\textbackslash}result}}
\newcommand{\separated}{\keyword{{\textbackslash}separated}}
\newcommand{\sizeof}{\keyword{sizeof}}
\newcommand{\strlen}{\keyword{{\textbackslash}strlen}}
\newcommand{\Sum}{\keyword{sum}}
\newcommand{\valid}{\keyword{{\textbackslash}valid}}
\newcommand{\validrange}{\keyword{{\textbackslash}valid\_range}}
\newcommand{\offset}{\keyword{{\textbackslash}offset}}
\newcommand{\blocklength}{\keyword{{\textbackslash}block\_length}}
\newcommand{\baseaddr}{\keyword{{\textbackslash}base\_addr}}
\newcommand{\comparable}{\keyword{{\textbackslash}comparable}}

\newcommand{\N}{\ensuremath{\mathbb{N}}}
\newcommand{\ra}{\ensuremath{\rightarrow}}
\newcommand{\la}{\ensuremath{\leftarrow}}


% BNF grammar

\newif\ifspace
\newif\ifnewentry
\newcommand{\addspace}{\ifspace \; \spacefalse \fi}
\newcommand{\term}[1]{\addspace\hbox{\texttt{#1}} \spacetrue}
\newcommand{\nonterm}[1]{%
\addspace\hbox{\textsl{#1}\ifnewentry\index{#1@\textsl{#1}!non-terminal}\fi}\spacetrue}
\newcommand{\repetstar}{^*\spacetrue}
\newcommand{\repetplus}{^+\spacetrue}
\newcommand{\repetone}{^?\spacetrue}
\newcommand{\lparen}{\addspace(}
\newcommand{\rparen}{)}
\newcommand{\orelse}{\addspace\mid\spacetrue}
\newcommand{\sep}{ \\[2mm] \spacefalse\newentrytrue}
\newcommand{\newl}{ \\ & & \spacefalse}
\newcommand{\alt}{ \\ & \mid & \spacefalse}
\newcommand{\is}{ & ::= & \newentryfalse}
\newenvironment{syntax}{$$\begin{array}{rrll}\spacefalse}{\end{array}$$}
\newcommand{\synt}[1]{$\spacefalse#1$}
\newcommand{\emptystring}{\epsilon}
\newcommand{\below}{See\; below}

% colors

\definecolor{darkgreen}{rgb}{0, 0.5, 0}

% theorems

\newtheorem{example}{Example}[chapter]

%%% Local Variables:
%%% mode: latex
%%% TeX-PDF-mode: t
%%% TeX-master: "main"
%%% End:


\setlength{\textheight}{240mm}
\setlength{\topmargin}{-10mm}
\setlength{\textwidth}{160mm}
\setlength{\oddsidemargin}{0mm}
\setlength{\evensidemargin}{0mm}

\renewcommand{\textfraction}{0.01}
\renewcommand{\topfraction}{0.99}
\renewcommand{\bottomfraction}{0.99}

\usepackage{fancyheadings}
\pagestyle{fancyplain}
\renewcommand{\footrulewidth}{0.4pt}
\addtolength{\headheight}{2pt}
\addtolength{\headwidth}{1cm}
\renewcommand{\chaptermark}[1]{\markboth{#1}{}}
\renewcommand{\sectionmark}[1]{\markright{\thesection\ #1}}
\lhead[\fancyplain{}{\bfseries\thepage}]{\fancyplain{}{\bfseries\rightmark}}
\chead{}
\rhead[\fancyplain{}{\bfseries\leftmark}]{\fancyplain{}{\bfseries\thepage}}
\lfoot{\fancyplain{}{ANSI C Specification Language}}
\cfoot{}
\rfoot{\fancyplain{}{CAT RNTL project}}

\begin{document}
\sloppy

\begin{titlepage}
\begin{center}
~\vfill

\includegraphics[height=60mm]{FramaC.jpg}

\vspace{20mm}

{\Huge\bfseries ACSL: ANSI C Specification Language}

\vspace{20mm}

{P.~Baudin, J.-C.~Filli\^atre, Th.~Hubert,
  C.~March\'e, B.~Monate, Y.~Moy, V.~Prevosto}

\vspace{20mm}

Version of \today

\vfill 

\includegraphics[height=14mm]{cealistlogo.jpg}
\hfill 
\includegraphics[height=12mm]{inriafuturslogoshort.jpg}

\end{center}
\end{titlepage}

\clearpage
\label{chap:contents}
\tableofcontents

\chapter{Introduction}

This document is a reference manual for ACSL, an acronym for ``ANSI C
Specification Language''. This is a Behavioral Interface Specification
Language (aka BISL) implemented in the \textsc{Frama-C} framework. As
its name suggests it, it aims at specifying behavioral properties of C
source code. The main inspiration for this language comes from the
specification language of the \textsc{Caduceus}
tool~\cite{filliatre04icfem,filliatre07cav} for deductive verification
of behavioral properties of C programs. It is itself inspired from the
\emph{Java Modeling Language} (JML~\cite{leavens00jml}) which aims at
similar goals for Java source code: indeed it aims both at
\emph{runtime assertion checking} and \emph{static verification} using
the \textsc{ESC/Java2} tool~\cite{ESCJava2}, where we aim at
\emph{static verification} and \emph{deductive verification} (see
Appendix~\ref{sec:comp-jml} for a detailed comparison between ACSL and
JML).

Going back further in history, JML design was guided by the general
\emph{design-by-contract} principle proposed by Bertrand Meyer, who
took his own inspiration from the concepts of preconditions and
postconditions on a routine, going back at least to Dijkstra, Floyd and
Hoare in the late 60's and early 70's, and originally implemented in
the \textsc{Eiffel} language.

In this document, we assume that the reader has a good knowledge of
the ANSI C programming language~\cite{KR88,standardc99}.

\section{Organization of this document}

In this preliminary chapter we introduce some definitions and
vocabulary, and discuss generalities about this specification
language.  Chapter~\ref{chap:base} presents the specification language
itself.  Chapter~\ref{chap:lib} presents additional informations about
\emph{libraries} of specifications.  Chapter~\ref{chap:appendix}
finally provides a few additional information.  A detailed table of
contents is given on page~\pageref{chap:contents}

\section{Glossary}

\begin{description}
\item[pure expressions] \index{pure expression} In ACSL setting, a
  \emph{pure} expression is a C expression which contains no assignments, no
  incrementation operator \verb|++| or \verb|--|, no function call,
  and no access to a volatile object. The set of pure expression is a
  subset of the set of C expressions without side effect (C
  standard~\cite{KR88,standardc99}, \S 5.1.2.3, alinea 2).

\item[left-values] \index{left-value}\index{lvalue} 

  A \emph{left-value} (\emph{lvalue} for short) is an expression which
  denotes some place in the memory during program execution, either on
  the stack, on the heap, or in the static data segment. It can be
  either a variable identifier or an expression of the form $*e$,
  $e[e]$, $e\verb|.|id$ or $e\verb|->|id$, where $e$ is any expression
  and $id$ a field name. See C standard, \S 6.3.2.1 for a more
  detailed description of lvalues.

  A \emph{modifiable lvalue} is an lvalue allowed in the left part of
  an assignment.
 
\item[pre-state and post-state]
    \index{pre-state}\index{post-state}
    
    For a given function call, the \emph{pre-state} denotes the
    program state at the beginning of the call, including the
    current values for the function parameters. the \emph{post-state}
    denotes the program state at the return of the call.

\item[function behavior] \index{function behavior} \index{behavior}

  A \emph{function behavior} (\emph{behavior} for short) is a set of
  properties relating the pre-state and the post-state for a
  possibly restricted set of pre-states (behavior \emph{assumptions}).

\item[function contract] \index{function contract} \index{contract} 

  A \emph{function contract} (\emph{contract} for short) forms a
  specification of a function, consisting of the combination of a
  precondition (a requirement on the pre-state for any caller to that
  function), a collection of behaviors, and possibly a measure in case
  of a recursive function.

\end{description}

\oldremark{Claude}{ce document doit etre illustr� par des
exemples. Toute construction qui ne serait pas illustr�e par un
exemple sera non-retenue.}

\oldremark{Patrick}{utiliser en priorit� les constructions de JML,
ensuite les g�n�raliser lorsque cela a un sens, et en dernier recours,
en cr�er de nouvelles. Il y a peut-�tre des constructions
du C++ ou C\# � utiliser ou � g�n�raliser.
Idem avec les extensions de GCC.}

\section{Generalities about Annotations}

In this document, we consider that specifications are given as annotations 
in comments written directly in C source files, so that source files remain
compilable\footnote{Other means of attaching annotations to source
  files, without modifying them, are left to user tools.}. Those
comments must start by \verb|/*@| or \verb|//@| and end as usual in~C.

\subsection{Kinds of annotations}

\begin{itemize}
\item Global annotations:
  \begin{itemize}
  \item function contract. Such an annotation is inserted just before
    the declaration or the definition of a function.
    See section~\ref{sec:fn-behavior}.

  \item global invariant. This is allowed at the level of global declarations.
    See section~\ref{sec:invariants}.

  \item type invariant. This allows both structure or union
    invariants, and invariants on type names introduced by \typedef.
    See section~\ref{sec:invariants}.

  \item logic specifications: logic type introduction, introduction
    or definition of logic functions or predicates, axioms. Such an
    annotation is placed at the level of global declarations.

  \end{itemize}

\item Statement annotations:
  \begin{itemize}
  \item \assert clause. These are allowed
    everywhere a C label is allowed, or just before a
    block closing brace.  
    \oldremark{Claude}{About \assume{} clauses: current
      discussion is that it is not considered as an element of
      specification, so not present here. This should be part of proof
      management done by tools.} 
    \oldremark{Patrick}{En C, seules les
      instructions peuvent �tre �tiquett�es. GCC �tend cela aux
      accolades fermantes des blocs, y compris celle fermant le corps
      de fonction. On peut dire que l'on fait de m�me lorsque l'on
      place des annotations juste avant la fermerture d'un bloc.}
    
  \item loop annotation (invariant, variant, assign clauses) is
    allowed immediately before a loop statement: \For, \While,
    \Do\ldots \While. See Section~\ref{sec:loop_annot}

  \item statement contract. Very similar to a function contract, and
    placed before a statement or a block.  Semantical condition must
    be checked (normal termination only, no goto going inside, no goto
    going outside).  See Section~\ref{sec:statement_contract}
    \oldremark{Patrick}{as-t'on droit au \old dans le \ensures de
      cette annotation~? Oui, pour refer a l'etat avant le statement
      consider\'e}
    
  \item ghost code: is regular C code, only visible from the
    specification, that is only allowed to modify ghost variables. See
    section~\ref{sec:ghost}. This includes ghost braces for enclosing blocks.

  \end{itemize}

%\item Attribute annotations: \experimental. See
%  Section~\ref{sec:attribute_annot}.

\end{itemize}

\subsection{Parsing annotations in practice}

In JML, parsing is done by just ignoring \verb|//@|, \verb|/*@| and
\verb|*/| and the level of lexing. This technique could modify the
semantics of the C code, for example: \input{annot1.pp}

In our language this is forbidden. Technically, the current
implementation of Frama-C isolates the comments in a first step of
syntax analysis, and then parses a second time. Nevertheless, the
grammar and the corresponding parser must be carefully designed to
avoid interaction of annotations with the code. For example, in such a
code: 
\input{annot2.pp} 
the statement \verb|c=1| must be understood as the \texttt{then}
branch of the \texttt{if}. This is ensured by the grammar below,
saying that \verb|assert| annotations are not statement themselves,
but attached to the statement that follows, like C labels.

\subsection{About preprocessing}

This document considers C source \emph{after} preprocessing. Tools
must decide what to do for preprocessing phase: what to do with
annotations, should macro substitution be performed or not, etc.

\subsection{About keywords}

Additional keywords of the specification language start with a
backslash, if they are used in position of a term or a predicate
(which are defined in the following).  Otherwise they do not start
with a backslash (like \ensures{}) and they remains valid identifiers.


\section{Notations for grammars}

In this document, grammar rules are given in BNF form. In grammar
rules, we use extra notations $e^*$ to denote repetition of zero, one
or more occurrences of $e$, $e^+$ for repetition of one or more
occurrences of $e$, $e^?$ for zero or one occurrence of $e$.



%%% Local Variables:
%%% mode: latex
%%% TeX-PDF-mode: t
%%% TeX-master: "main"
%%% End:



\chapter{Specification language}
\label{chap:base}

\section{Logic expressions}
\label{sec:expressions}

This first section presents the language of expressions one can use in
the annotations. These are called below \emph{logic expressions}. They
corresponds more or less to pure C expressions, without function
calls, with additional constructs that we introduce one at a time.

\begin{figure}[p]
  \fbox{\begin{minipage}{0.97\textwidth} \begin{syntax}
  literal ::= "\true" | "\false" ; boolean constants
       | integer ; integer constants (lexical token)
       | real ; real constants (lexical token)
       | string ; string constants (lexical token)
       | character ; character constants (lexical token)
       \ 
  bin-op ::= "+" | "-" | "*" | "/" | "%" | "<<" | ">>";
       | "==" | "!=" | "<=" | ">=" | ">" | "<" ;
       | "&&" | "||" | "^^"  ; boolean operations
       | "&" | "|" | "-->" 
       | "<-->" | "^"; bitwise operations
       \
  unary-op ::= "+" | "-" ; unary plus and minus
       | "!" ; boolean negation
       | "~" ;  bitwise complementation
       | "*" ; pointer dereferencing
       | "&" ; address-of operator
       \
  term ::= literal ; literal constants
       | poly-id ; variables
       | unary-op term ;
       | term bin-op term ;
       | term "[" term "]" ; array access
       | "{" term "\with" "[" term "]" "=" term "}" ; array functional modifier
       | term "." id  ; structure field access
       | "{" term "\with" "."id "=" term "}" ; field functional modifier
       | term "->" id ;
       | "(" type-expr ")" term  ; cast
       | poly-id "(" term ("," term)* ")" ; function application
       | "(" term ")" ; parentheses
       | term "?" term ":" term ; ternary condition
       | "\let" id "=" term ";" term ; local binding
       | "sizeof" "(" term ")" ;
       | "sizeof" "(" C-type-expr ")" ;
       | id ":" term ; syntactic naming
       | string ":" term ; syntactic naming
       \
  poly-id ::= ident ; 
       \
  ident ::= id ;  lexical identifier token 
\end{syntax}

%%% Local Variables:
%%% mode: latex
%%% TeX-master: "main"
%%% End:

    \end{minipage}}
  \caption{Grammar of terms and predicates}
\label{fig:gram:term}
\end{figure}

Figure~\ref{fig:gram:term} presents the grammar for logic expressions.
In that grammar, we distinguish between \emph{predicates} and
\emph{terms}, following the usual distinction between propositions and
terms in classical multi-sorted first-order logic.


The additional constructs are as follows:
\begin{itemize}
\item C operators \verb|&&|, \verb+||+ and \verb|!| are used as
  logical connectives. There are additional connectives \verb|==>| for
  implication, \verb|<==>| for equivalence, \verb|^^| for exclusive
  or.

\item Quantifications: universal $\Forall \tau x; e$ and existential
  $\Exists \tau x; e$

\item Local naming $\Let x = e_1 ; e_2$ introduce the name $x$ for
  expression $e_1$ which can be used in expression $e_2$.

\item conditional $\If c \Then e_1 \Else e_2$. There is a subtlety
  here: the condition may be either a boolean term or a predicate.  In
  case of a predicate, the two branches must be also predicates, so
  that this construct acts as a connective with the following
  semantics: $\If c \Then e_1 \Else e_2$ is equivalent to $(c
  \verb|==>| e_1) \verb|&&| (\verb|!| c \verb|==>| e_2)$

\item syntactic naming: $id \verb|::| e$ is a term or a predicate
  equivalent to $e$. It is different from local naming with $\Let$:
  the name cannot be reused in other terms of predicates. It is only
  for tools purposes.

\item The construct $t_1~relop_1~t_2~relop_2~t_3 \cdots t_k$ is a
  shortcut for $t_1~relop_1~t_2 ~\verb|&&|~ t_2~relop_2~t_3
  ~\verb|&&|~ \cdots $.

\item Function and predicate application are not applications of C functions, but of logic functions or predicates: see Section~\ref{sec:logicspec}
\end{itemize}

\subsection{Semantics, typing}

always defined (2-valued logic). Distinction entre booleens et
predicats.  TODO: referer aux travaux de P Chalin ? 
Donner des examples

on veut la logique equationnelle classique, cad que l'axiome
\[
\forall x, x=x
\]
est valide. Donc on ne peut pas introduire de construction
non-d�terministe comme  $(\texttt{any} x \mid P)$


Types de la logique (see Section~\ref{sec:logicspec}:
\begin{itemize}
\item types mathematiques: real, integer, boolean
\item types du C
\item types logiques introduits par l'utilisateur
\end{itemize}

Typing rules: TODO

\section{Function contracts}
\label{sec:fn-behavior}



\begin{figure}[t]
  \fbox{\begin{minipage}{0.97\textwidth}
      \begin{syntax}
  function-contract ::= requires-clause* terminates-clause? decreases-clause? ;
               simple-clauses* named-behavior* 
  \
  requires-clause ::= "requires" predicate ";"
  \
  terminates-clause ::= "terminates" pred ";"
  \
  decreases-clause ::= "decreases" term ("for" ident)? ";"
  \
  simple-clauses ::= assigns-clause | ensures-clause | {abrupt-clause-fn}
  \
  assigns-clause ::= "assigns" locations ";"
  \
  locations ::= location ("," location) * | "\nothing"
  \
  ensures-clauses ::= "ensures" predicate ";"
  \
  named-behavior ::= "behavior" id ":" behavior-body
  \
  behavior-body ::= assumes-clause*
                     {requires-clause}* 
                    simple-clauses* \
  assumes-clause ::= "assumes" predicate ";"
\end{syntax}

    \end{minipage}}
    \caption{Grammar of function contracts}
  \label{fig:gram:contracts}
\end{figure}

\begin{figure}[t]
  \fbox{\begin{minipage}{0.97\textwidth}
      \begin{syntax}
  term ::= "\old" "(" term ")" ; old value
       | "\result" ; result of a function
       | "\at" "(" term "," id ")" 
       \
  pred ::= "\old" "(" pred ")" ;
       | "\at" "(" pred "," id ")" 
\end{syntax}

%%% Local Variables:
%%% mode: latex
%%% TeX-master: "main"
%%% End:

    \end{minipage}}
    \caption{\old and \result in terms}
  \label{fig:gram:oldandresult}
\end{figure}



The Figure~\ref{fig:gram:contracts} show a grammar for the function
contracts. The grammars for non-terminal locations is given later,
informally, they denotes C l-values. We also introduce additional constructs 
for terms as given on figure~\ref{fig:gram:oldandresult}. These are the following:
\begin{itemize}
\item $\old(e)$ denotes the value of $e$ in the pre-state of the function. 
\item \result{} denotes the returned value of the function.
\end{itemize}


They both can be used only in \ensures{} clauses.


\remark{Patrick}{pourquoi seuls les ``behavior'' peuvent avoir un
  nom~? Ne peut-t'on pas avoir le ``ident:'' optionnel pour chacune
  des clauses~? Claude: parce que le nommage de behavior doit servir
  pour les associer a des clauses du corps de la fonction, p. ex.
  assert ou lopp invariant. Mais on peut aussi proposer par ailleurs
  de nommer des predicate ou terms, p. ex. avec la notation Caduceus
  (id :: f) }

\subsection{Simple function contracts}

A simple function contract, having no named behavior, as the following
form:
\begin{flushleft}\ttfamily
/*@ requires $P_1$; \\
~~@ requires $P_2$; \\
~~@ assumes $A_1$; \\
~~@ assumes $A_2$; \\
~~@ assigns $L_1$; \\
~~@ assigns $L_2$; \\
~~@ ensures $E_1$; \\
~~@ ensures $E_2$; \\
~~@ decreases $m$ for $R$; \\
~~@r*/
\end{flushleft}

The semantics of such a contract is as follows:
\begin{itemize}
\item The caller of the function must guarantee that it is called in a
  state where the property $P_1 \verb|&&| P_2$ holds.
\item The called function returns a state where the property
  $\old(A_1) \verb|&&| \old(A_2) \verb|==>| E_1 \verb|&&| E_2$ holds. 
\item If the function is called in a pre-state where
  $A_1 \verb|&&| A_2$ holds, then all memory locations of that
  pre-state that do not belong to the set $L_1 \cup L_2$ remain allocated
  and are left unchanged in the post-state.
\item If the decreases clause is present, the function can call only
  functions (including itself) which have also a decreases clause with
  the same relation $R$, which are called in a state for which the
  measure expression $m$ is smaller.
\end{itemize}

Notice that the \assumes{} clauses, and the multiplicity of clauses in
general, are proposed mainly to improve readibility (and to avoid \old and
some duplications of formulas), since the contract above is equivalent
to the following simplified one:
\begin{flushleft}\ttfamily
/*@ requires $P_1 \verb|&&| P_2$; \\
~~@ assigns $L_1,L_2$; \\
~~@ ensures $\old(A_1) \verb|&&| \old(A_2) \verb|==>| E_1 \verb|&&| E_2$; \\
~~@ decreases $m$ for $R$; \\
~~@*/ 
\end{flushleft}

\begin{example}
The following function is given a simple contract for computation of integer square root, rounded to the floor.
\input{isqrt.pp}

The contract means that the function must not be called with a
negative argument, and in return the result satisfies the conjunction
of the three predicates given in \ensures{} clauses.
\end{example}

\begin{example}
The following function is given a contract to specify it increments the value pointed by the pointer given as argument.
\input{incrstar.pp}

The contract means that the function must be called with a pointer $p$
that points to a safely allocated memory location (See
Section~\ref{sec:pointers} for details on $\valid$ built-in predicate). It
modifies only the value pointed by $p$, and more precisely it
increments it by one.
\end{example}


\subsection{Contracts with named behaviors}

More generally, a function contract as the following form with named
behaviors (restricted to two behaviors for readability):
\begin{flushleft}\ttfamily
/*@ requires $P$; \\
~~@ behavior $b_1$: \\
~~@ ~~requires $R_1$; \\
~~@ ~~assumes $A_1$; \\
~~@ ~~assigns $L_1$; \\
~~@ ~~ensures $E_1$; \\
~~@ behavior $b_2$:  \\
~~@ ~~requires $R_2$;  \\
~~@ ~~assumes $A_2$; \\
~~@ ~~assigns $L_2$; \\
~~@ ~~ensures $E_2$; \\
~~@ decreases $m$ for $R$; \\
~~@r*/
\end{flushleft}

The semantics of such a contract is as follows:
\begin{itemize}
\item The caller of the function must guarantee that it is called in a
  state where the property $P \verb|&&| (R_1 \verb+||+ R_2)$
  holds.
\item The called function returns a state where the properties
  $\old(R_i) \verb|&&| \old(A_i) \verb|==>| E_i$ hold for each $i$.
\item for each $i$, if the function is called in a pre-state where
  $R_i \verb|&&| A_i$ holds, then all memory locations of that
  pre-state that do not belong to the set $L_i$ remain allocated
  are left unchanged in the post-state.
\item If the decreases clause is present, the function can call only
  functions (including itself) which has also a decreases clause with
  the same relation $R$, which are called in a state for which the
  measure expression $m$ is smaller.
\end{itemize}

Notice that the assumes clauses and the requires clause in the
behaviors are proposed mainly to improve readibility (to avoid \old
and some duplications of formulas), since the contract above is
equivalent to the following simplified one:
\begin{flushleft}\ttfamily
/*@ requires $P_1$ \&\& $P_2$ \&\& ($R_1$ || $R_2$); \\
~~@ behavior $b_1$: \\
~~@ ~~assigns $L_1$; \\
~~@ ~~ensures \old($R_1$) \&\& \old($A_1$) ==> $E_1$ \\
~~@ behavior $b_2$:  \\ 
~~@ ~~assigns $L_2$; \\
~~@ ~~ensures \old($R_2$) \&\& \old($A_2$) ==> $E_2$; \\
~~@ decreases $m$ for $R$; \\
~~@*/ 
\end{flushleft}


\begin{example}
In the following, \texttt{bsearch($t,n,v$)} searches for element $v$ in array $t$ between index $0$ and $n-1$.
\input{bsearch.pp}

The intention is to perform a binary search, which requires that the
array $t$ is sorted in increasing order: this is the purpose of the
predicate named \texttt{t\_is\_sorted} in the precondition. The
remaining of the precondition is to require that the array is safely
allocated for at least the index from $0$ to $n-1$. The two named
behaviors correspond respectively to the succesful behavior and the
failing behavior.

See~\ref{sec:loop_invariant} for a continuation of this example.
\end{example}

\begin{example}
  The following function illustrates the importance of different
  \assigns{} clauses for each behavior.  

  \input{cond_assigns.pp} 

  Its contract means that it assigns values pointed by $p$ or by $q$,
  conditionally on the sign of $n$.
\end{example}


\subsection{Remarks}

\remark{Yannick}{JML propose deux modes de sp�cification des fonctions~:
 soit � base de ``requires'' et ``ensures'', soit � base de
 ``behavior''. JML n'a pas de ``assumes'' dans les
  ``behaviors'', mais des ``requires''. Dans le cas o� les deux modes
  peuvent se combiner, la s�mantique est la suivante~:}

\begin{verbatim}
  requires P_1
  requires P_2
  ensures  Q_1
  ensures  Q_2
  behavior x_1: requires R_1 ensures E_1
  behavior x_2: requires R_2 ensures E_2
\end{verbatim}


\begin{verbatim}
 pre-condition : P_1 and P_2
             and (R_1 or R_2)
 post-condition: Q_1 and Q_2
             and (\old(R_1) implies E_1)
             and (\old(R_2) implies E_2)
\end{verbatim}

\remark{Patrick}{les ``behavior'' de JML pr�sente l'avantage de
  pouvoir sp�cifier une fonction par cas (non exclusifs)
  et de v�rifier que les cas d'appel sont sp�cifi�s (sinon on ne peut
  v�rifier la pr�-condition).
  Il semble important � Airbus de pouvoir s'assurer qu'ils ont
  �nonc� l'ensemble des cas.}

\remark{Yannick}{avoir ``requires'' en plus des ``assumes'' dans les
  ``behaviors'' semble utile :
la s�mantique consiste � rajouter $(A_i \ra R_i)$ en conjonction de la
pr�condition globale}

\begin{verbatim}
  requires P_1
  requires P_2
  ensures  Q_1
  ensures  Q_2
  behavior x_1: requires R_1 assumes A_1 ensures E_1
  behavior x_2: requires R_2 assumes A_2 ensures E_2
\end{verbatim}


\begin{verbatim}
 pre-condition : P_1 and P_2
             and (A_1 implies R_1)
             and (A_2 implies R_2)
 post-condition: Q_1 and Q_2
             and (\old(A_1) implies E_1)
             and (\old(A_2) implies E_2)
\end{verbatim}

\subsection{Memory locations}
\label{sec:locations}

There are several places where one needs to describe a set of memory locations: \assigns{} clauses, \reads{} clauses.

\begin{figure}
  \fbox{\begin{minipage}{0.97\textwidth}
      \begin{syntax}
  tset ::= "\empty" ; empty set
       | term ; singleton
       | {[restricted to singleton]tset "->" id} ;
       | {[restricted to singleton]tset "." id} ;
       | {[restricted to singleton] "*" tset} ;
       | {[base is restricted to a singleton]tset "[" tset "]"} ;
       | {[can only appear in position of index]term? ".." term?} ; range
       | "\union" "(" tset ("," tset)* ")" ; union of locations
       | "\inter" "(" tset ("," tset)* ")" ; intersection
       | {[only in the form term+term, term+range or range+term]tset "+" tset} ;
       | "(" tset ")" ;
       | "{" tset "|" binders (";" pred)? "}" ; set comprehension
       \
  pred ::= {"\subset" "(" tset "," tset ")"} ; set inclusion
\end{syntax}

%%% Local Variables:
%%% mode: latex
%%% TeX-master: "main"
%%% End:
  
    \end{minipage}}
  \caption{Grammar for memory locations}
\label{fig:gram:locations}
\end{figure}

The grammar for denoting such a set of memory locations is given
Figure~\ref{fig:gram:locations}.

The semantics is given as follows, where $s$ denotes any tset:
\begin{itemize}
\item a simple term denotes a singleton
\item $s\ra id$ denotes the set of $x\ra id$ for each $x \in s$
\item $s.id$ denotes the set of $x.id$ for each $x \in s$
\item $*s$ denotes the set of $*x$ for each x in s
\item $s_1[s_2]$ denotes the set of $x_1[x_2]$ for each $x_1 \in s_1$
  and $x_2 \in s_2$
\item $t_1 .. t_2$ denotes the set of integers between $t_1$ and
  $t_2$, included
\item $s_1,s_2$ denotes the union of $s_1$ and $s_2$
\item $s_1+s_2$ denotes the set of $x_1+x_2$ for each $x_1 in s_1$ and $x_2 in s_2$
\item $(s)$ denotes the same set as $s$
\item $\{ s \mid b ; P \}$ denotes set comprehension: set of term denoted by $s$ for each values of binders satisfying predicate $P$. Binders $b$ are bound in $s$ and $P$
\end{itemize}

A \emph{location} is any set of terms denoting a set of lvalues.
Only locations are valid as argument to \assigns{} clauses

Examples:
\begin{c}
assigns  (forall struct list *p ; reachable(root,p)) -> hd
\end{c}


\remark{Patrick}{ne faut-t'il pas �tendre les clauses ``assigns'' aux
  clauses ``from'' de CAVEAT prenant en comptes les locations lues, et les
  expressions fonctionnelles~?}

\subsection{Typing rules}

Two judgements:
\begin{itemize}
\item $\Gamma,\Lambda \vdash e : loc \tau$ means e is a set of location of type
  $tau$
\item $\Gamma,\Lambda \vdash e : tset \tau$ means e is a set of terms of type
  $tau$
\end{itemize}
$\Gamma$ is the C environment and $\Lambda$ is the logic environment.

Rules:
\[
\frac{\vdash e:loc \tau}{\vdash e: tset \tau}
\]
\[
\frac{\tau x \in \Gamma}{\vdash x: loc \tau}
\]
\[
\frac{e:tset \tau*}{\vdash *e: loc \tau}
\]
\[
\frac{e_1:tset \tau \quad e_2:tset \tau}{\vdash e_1,e_2: \tau}
\]
\[
\frac{e: tset struct S* \quad e_2:tset \tau}{\vdash e->f : loc \tau}
\]
\[
\frac{b\cup \Lambda \vdash e: tset \tau}{\vdash \{ e \mid b ; P \} : tset \tau }
\]
idem for $loc \tau$

Notes:
\begin{itemize}
\item Quantification can be be made over any type (both C and
  logic types).
\item Quantification over pointers, structures, union, etc.
  are possible too. TODO: the meaning must be carefully defined.
\end{itemize}

\section{Statement annotations}


  \begin{itemize}
  \item \assert, label
    \remark{claude}{permettre de nommer les asserts}
  \item loop invariants (+loop assigns ?)
  \item block behaviors (normal termination only, no goto going
    inside, no goto going outside)
  \end{itemize}

\subsection{Assertions and logical labels} 

\begin{syntax}
  statement ::= "/*@" logical-label "*/" statement \
  logical-label ::= "label" id: ;
  | "assert" pred ";" ;
  | "assert" id ":" pred ";" ;
  | "assert" "for" id ":" pred ";" ;
\end{syntax}

The \texttt{label} id annotation simply allows one to refer to the value of some expression $e$ in the state before the statement, in further annotations, using the constructs \at(e,id)

The \texttt{assert} p clause means that p must hold in the state
before the statement.

The last two constructs allows to give a name to the assertion. In the
case \texttt{assert for} id, id must be a behavior identifier for the
current function. It means that this assertion is meant to be valid
only for the considered behavior.

\remark{Claude}{unifier les 2 derniers cas? car a-t-on besoin de
  nommer un assert en dehors d'un behavior de la fonction?}



TODO: $\at(e,label)$


\subsection{Statement contracts}

TODO

\subsection{Loop annotations}
\label{sec:loop_invariant}

TODO

semantics of loop invariants: in particular for {\tt for} loops

\begin{syntax}
  invariant-clause ::= "loop" "invariant" predicate \
\end{syntax}

loop assigns: semantics?


\section{Pointers and physical adressing}
\label{sec:pointers}

\subsection{Memory blocks and pointer dereferencing}

\begin{itemize}
\item \baseaddr{} base address of an allocated pointer
\[
\baseaddr{} : \alpha {\tt *} \ra {\tt char *}
\]
\item \blocklength{} length of the allocated block of a pointer
\[
\blocklength{} : \alpha {\tt *} \ra {\tt size\_t}
\]

\end{itemize}

Shortcuts:
\begin{itemize}
\item offset(p) returns the offset between p ans its base address

  \begin{eqnarray*}
    offset &:& \alpha {\tt *} \ra {\tt size\_t}  \\
    offset(p) &=& (char*)p - \baseaddr(p)
  \end{eqnarray*}

\item valid(p) tells whether dereferencing p is safe

  \begin{eqnarray*}
    valid : \alpha {\tt *} \ra {\tt boolean} \\
    valid(p) = offset(p) \geq 0 \land offset(p) + sizeof(*p) \leq \blocklength(p)
  \end{eqnarray*}

\item \comparable{} (checks whether two pointers are comparable as defined
  in the ANSI standard: TODO by benjamin)
\[
\comparable{} : \alpha {\tt *} \ra \beta {\tt *} \ra {\tt boolean}
\]
\end{itemize}


\subsection{Separation}

pointer separation :
\[
p \not\equiv q := \baseaddr(p) \neq \baseaddr(q) \lor |(char*)p - (char*)q| \geq \max(\sizeof(p),\sizeof(q))
\]

$\separated(loc1,..,loc_n)$ : means that for if $i\neq j$, if $x\in loc_i$ and $y \in loc_j$ then $\&x \not\equiv \& y$

where each $loc_i$ is a set of memory location as defined in Section~\ref{sec:locations}.

\subsection{Allocation and deallocation}

\experimental

\begin{itemize}
\item built-in predicate \fresh, specifying in a post-condition that a
  pointer was not allocated in the pre-state.
 
\item built-in predicate \freed, specifying in a post-condition that a
  pointer was allocated in the pre-state but not anymore.
\end{itemize}

\section{Termination}

Property of termination concerns both loops and recursive function calls. 
For that purpose, loops can be annotated with \emph{loop variants}\index{loop variant}, and functions can be annotated with such variants too.

\subsection{integer measures}

For loops:
\begin{syntax}
variant-clause ::= "loop variant" e 
\end{syntax}

For functions:
\begin{syntax}
decreases-clause ::= "decreases" e 
\end{syntax}
where $e$ has type integer

\subsection{general measures}

More general case of measures on other types: use the keyword for:

\begin{syntax}
variant-clause ::= "loop variant" e "for" id \
decreases-clause ::= "decreases" e "for" id
\end{syntax}

where $e$ has type $\tau$ and $id$ is a logic predicate of type $\tau,\tau \ra
prop$ (see Section~\ref{sec:logicspec}


\section{Logic specifications}
\label{sec:logicspec}

\begin{figure}[t]
  \fbox{\begin{minipage}{0.97\linewidth} \begin{syntax}
  C-global-decl ::= "/*@" logic-decl{[Only one declaration per annotation]+} "*/"
  \
  logic-decl ::= logic-type-decl ;
          | logic-const-decl | logic-const-def ;
          | logic-predicate-decl | logic-predicate-def ;
          | logic-function-decl | logic-function-def ;
          | lemma-def
  \
  type-var ::= id
  \
  type-var-binders ::= "<" type-var (, type-var)* ">"
  \
  logic-type-decl ::= "type" logic-type ";" ;
  \
  logic-type ::= id ;
  | id type-var-binders ; polymorphic type
  \
  logic-type-expr ::= type-var ; type variable
  | id "<" type-expr (, type-expr)* ">" ; polymorphic type
  \
  poly-id ::= id ; normal identifier
  | id type-var-binders ; identifier for polymorphic object
  \
  logic-const-def ::= {"logic" type-expr poly-id "=" term ";"}
  \
  logic-function-def ::= "logic" type-expr poly-id parameters "=" term ";"
  \
  logic-predicate-def ::= "predicate" poly-id parameters "=" pred ";"
  \
  logic-const-decl ::= { "logic" type-expr poly-id ";"}
  \
  logic-function-decl ::= "logic" type-expr poly-id parameters ";"
  \
  logic-predicate-decl ::= "predicate" poly-id parameters? ";"
  \
  parameters ::= "(" parameter (, parameter)* ")"
  \
  parameter ::= type-expr variable-ident
  \
  lemma-def ::= "axiom" poly-id ":" pred ";" ;
  | "lemma" poly-id ":" pred ";" ;
\end{syntax}
    \end{minipage}}  
  \caption{Grammar for logic declarations}
\label{fig:gram:logic}
\end{figure}

\begin{itemize}
\item logic specifications: introduction of new logic types,
  constants, functions, predicates and axioms.

logic types can be polymorphic: for example
\input{listdecl.pp}

\item recursive definitions: are allowed in logic function and predicate definitions. Example:  
  \input{max_index.pp}

mutual recursion

\item predefined logic specifications can be provided as libraries (see section~\ref{sec:speclibraries}, and imported using
\input{import.pp}

\item higher-order functions: $\lambda$, \Sum, \Max, \numof
\item {\tt enum}, recursive types


\item hybrid functions and predicates: take both C types and logic
  types as arguments.
\item model variables
\end{itemize}


\section{Abnormal termination}

\begin{itemize}
\item C function \verb|exit(n)|. special behavior notation
  \begin{verbatim}
    exit_behavior
       assumes
       assigns
       ensures ...
  \end{verbatim}
    where in ensures clause, \result is bound to n
\end{itemize}


\section{Dependencies information}

\experimental

\begin{verbatim}
assigns Locs from Locs
\end{verbatim}

\section{Functional expressions}

\experimental

trouver une syntaxe pour introduire les noms des fonctions implicites
derriere la construction
\begin{verbatim}
assigns Locs from Locs
\end{verbatim}

\section{Data invariants}
\label{sec:invariants}

\experimental

\begin{itemize}
\item Global invariants: invariants on global variables
\item Type invariants: invariants on struct, union, typedef
\item \verb|initially|: predicates that should hold after global
  variables initialization
\end{itemize}

Examples:
\input{invariants.pp}

\subsection{Semantics}

an invariant is true when ...\cite{barnett04jot}


\section{Ghost variables and statements}
\label{sec:ghost}

\experimental

\remark{Patrick}{les ``ghost statements'' correspondent � des
instructions d'observations des variables du C. Ces instructions
ne peuvent pas modifier les variables du C, mais que les ``ghost
variables''. Cela permet d'�crire plus facilement l'observateur
ad�quat � la preuve des propri�t�s puisque les propri�t�s peuvent
porter � la fois sur les variables du C et les  ``ghost variables''.
}


\begin{syntax}
  statement ::= ghoststatements statement \\
  ghoststatements ::= "/*@" "ghost" statement+ "*/"
\end{syntax}

%%% Local Variables:
%%% mode: latex
%%% TeX-master: "main"
%%% End:



\section{Module constructions}

\experimental

how to encapsulate several functions...

\section{Arithmetic, Overflow}

quantification can be either on mathematical \verb|integer| or bounded
types \verb|short|, \verb|char|, etc.

we need macros \verb|\max_range|, \verb|min_range| taking a C integer
type as argument, e.g. \verb|\max_range(unsigned char) = 255|\\

\remark{Patrick}{Ces macros, types et variables sont en principe
d�finies dans des \texttt{.h} que la norme sp�cifie en grande partie
(le nom l'est, le type peut y �tre contraint).
Il faudrait autant que possible se raprocher de ces noms.}


%%% Local Variables:
%%% mode: latex
%%% TeX-master: "main"
%%% End:


\chapter{Libraries}
\label{chap:lib}

This chapter is devoted to librairies of specification, built upon the ACSL specification language.

Section~\ref{sec:jessie} describes additional predicates propose by the Jessie plugin of Frama-C, to propose a slightly higher level of annotation.


%\section{Frama-C base library}
%
%\begin{itemize}
%\item \comparable{}: checks whether two pointers are comparable
%  as defined in the ANSI standard.
%  \[
%  \comparable{} : \verb|`a *| \ra \verb|`b *| \ra \boolean
%  \]
%\end{itemize}

\section{Jessie library: logical adressing of memory blocks}
\label{sec:jessie}

%Jessie is a plugin of the Frama-C platform, which connects to the
%corresponding jessie tool of the Why platform~\cite{filliatre07cav}
%for deductive verification of behavioral properties of programs.

The Jessie library is a collection of logic specifications whose
semantics is well-defined only on source codes free from
architecture-dependent features. In particular it is currently
incompatible with pointer casts or unions (although there is ongoing
work to support some of them~\cite{moy07ccpp}). As a consequence, in
this particular setting, a valid pointer of some type $\tau*$
necessarily points to a memory block which contains values of type
$\tau$.

\subsection{Abstract level of pointer validity}

The Jessie plugin currently assumes the input source code free from
architecture-dependent features. In particular it currently completely
disallows pointer casts or unions (although there is ongoing work to
support some of them~\cite{moy07ccpp}). As a consequence a valid
pointer of some type $\tau*$ necessarily points to a memory block
which contains values of type $\tau$. To model that, the jessie library introduce new logic functions:
\begin{flushleft}
\integer ~ \notimplemented{\offsetmin(`a *p)}; \\
\integer ~ \notimplemented{\offsetmax(`a *p)};
\end{flushleft}

\begin{itemize}
\item $\offsetmin(p)$ is the minimum integer $i$ such that $(p+i)$ is a
  valid pointer.

\item $\offsetmax(p)$: the maximum integer $i$ such that $(p+i)$ is a
  valid pointer
\end{itemize}
The following properties hold:
\begin{eqnarray*}
\offsetmin(p+i) &=& \offsetmin(p)-i \\
\offsetmax(p+i) &=& \offsetmax(p)-i
\end{eqnarray*}
It also introduce syntactic sugar:
\begin{eqnarray*}
\validrange(p,i,j) &:=& \offsetmin(p) <= i \land \offsetmax(p) >= j
\end{eqnarray*}
and the ACSL built-in predicate $\valid(p)$ is now equivalent to
$\validrange(p,0,0)$.

\subsection{Strings}

\experimental

The predicate
\[
\integer~\notimplemented{\strlen(char* p)}
\]
denotes the length of a 0-terminated C string. It is total function,
whose value is non-negative if and only if the pointer in argument is
really a string.

\begin{example}
  Here is a contract for the \verb|strcpy| function:
  \input{strcpyspec.pp}

\end{example}

\subsection{Field structures}

 "offsetof" "(" ... ")" ; \experimental

 "alignof" "(" C-type-expr ")" ; \experimental

\section{Memory leaks}

\experimental

Verification of absence of memory leak is outside the scope of the
specification language. On the other hand, various models could be set
up, using for example ghost variables.

\section{Libraries of logic specifications}
\label{sec:speclibraries}

A standard library is provided, in the spirit of the List module of
Section~\ref{sec:specmodules}


\subsection{Real numbers}
\label{sec:libreal}

A library of general purpose functions and predicate over real
numbers, floats and doubles.

Includes

\begin{itemize}
\item abs, exp, power, log, sin, cos, atan, etc. over reals

\item isFinite predicate over floats and doubles (means not NaN nor infinity)

\item rounding reals to floats or doubles with specific rounding modes.

\end{itemize}

\subsection{Finite lists}

\begin{itemize}
\item pure functions nil, cons, append, fold, etc.
\item Path, Reachable, isFiniteList, isCyclic, etc. on C linked-lists.
\end{itemize}


\subsection{Sets and Maps}

Finite sets, finite maps, in ZB-style.

\oldremark{PC}{

  proposer des syntaxes concretes pour des types logiques standards:
  ensemble en particulier, style ZB

}



%%% Local Variables:
%%% mode: latex
%%% TeX-PDF-mode: t
%%% TeX-master: "main"
%%% End:


\appendix

\chapter{Appendices}

\section{Quick reference card}

\todo{by Benjamin}

\section{Comparison with JML}

\todo{by Yannick}

\begin{verbatim}
  JML                  Frama-C

  loop_invariant       loop invariant 
  decreases            loop variant

\end{verbatim}

\remark{Yannick}{JML propose deux modes de sp�cification des fonctions~:
 soit � base de ``requires'' et ``ensures'', soit � base de
 ``behavior''. JML n'a pas de ``assumes'' dans les
  ``behaviors'', mais des ``requires''. Dans le cas o� les deux modes
  peuvent se combiner, la s�mantique est la suivante~:}

\begin{verbatim}
  requires P_1;
  requires P_2;
  ensures  Q_1;
  ensures  Q_2;
  behavior x_1: requires R_1; ensures E_1;
  behavior x_2: requires R_2; ensures E_2;
\end{verbatim}


\begin{verbatim}
 pre-condition : P_1 and P_2
             and (R_1 or R_2)
 post-condition: Q_1 and Q_2
             and (\old(R_1) implies E_1)
             and (\old(R_2) implies E_2)
\end{verbatim}


\remark{Yannick}{avoir ``requires'' en plus des ``assumes'' dans les
  ``behaviors'' semble utile :
la s�mantique consiste � rajouter $(A_i \ra R_i)$ en conjonction de la
pr�condition globale}

\begin{verbatim}
  requires P_1;
  requires P_2;
  ensures  Q_1;
  ensures  Q_2;
  behavior x_1: requires R_1; assumes A_1; ensures E_1;
  behavior x_2: requires R_2; assumes A_2; ensures E_2;
\end{verbatim}


\begin{verbatim}
 pre-condition : P_1 and P_2
             and (A_1 implies R_1)
             and (A_2 implies R_2)
 post-condition: Q_1 and Q_2
             and (\old(A_1) implies E_1)
             and (\old(A_2) implies E_2)
\end{verbatim}

\cleardoublepage
\addcontentsline{toc}{chapter}{\bibname}
\bibliographystyle{plain}
\IfFileExists{biblio-demons.tex}{%
\input{biblio-demons.tex}
}{\bibliography{./biblio}}

\cleardoublepage
\addcontentsline{toc}{chapter}{\listfigurename}
\listoffigures

\cleardoublepage
\addcontentsline{toc}{chapter}{\indexname}
\printindex

\end{document}

%%% Local Variables:
%%% mode: latex
%%% TeX-PDF-mode: t
%%% TeX-master: t
%%% End:
