\chapter{Librairies}
\label{chap:lib}

\section{Jessie library: logical adressing of memory blocks}

Definition: a \emph{valid} pointer is a pointer such that *p is properly
allocated.

\offsetmin, \offsetmax : $\alpha * \ra \N$

$\offsetmin(p)$: the minimum integer $i$ such that $(p+i)$ is a valid
pointer

$\offsetmax(p)$: the maximum integer $i$ such that $(p+i)$ is a valid
pointer

properties:

\begin{eqnarray*}
\offsetmin(p+i) &=& \offsetmin(p)-i \\
\offsetmax(p+i) &=& \offsetmax(p)-i
\end{eqnarray*}

syntactic sugar:

\begin{eqnarray*}
\validrange(p,i,j) &:=& \offsetmin(p) <= i \land \offsetmax(p) >= j \\
\valid(p) &:=& \validrange(p,0,0) 
\end{eqnarray*}

\remark{Benjamin}{propose plutot les noms ``indexmin'' et ``indexmax'' a la place de offsetmin et offsetmax}

\subsection{Strings}

\strlen

\remark{Yannick}{Peut disparaitre}

\subsection{Field structures}

 "offsetof" "(" ... ")" ; \experimental

 "alignof" "(" C-type-expr ")" ; \experimental

\section{Memory leaks}

\experimental

Verification of absence of memory leak is outside the scope of the
specification language. On the other hand, various models could be set
up, using for example ghost variables.


\section{Libraries of logic specifications}
\label{sec:speclibraries}

\subsection{Library list}

\begin{verbatim}
logic type 'a list = ...
\end{verbatim}

%%% Local Variables:
%%% mode: latex
%%% TeX-PDF-mode: t
%%% TeX-master: "main"
%%% End:
