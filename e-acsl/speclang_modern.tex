%; whizzy-master "main.tex"

\chapter{Specification language}
\label{chap:base}

%%%%%%%%%%%%%%%%%%%%%%%%%%%%%%%%%%%%%%%%%%%%%%%%%%%%%%%%%%%%%%%%%%%%%%%%%%%%%%%
%%%%%%%%%%%%%%%%%%%%%%%%%%%%%%%%%%%%%%%%%%%%%%%%%%%%%%%%%%%%%%%%%%%%%%%%%%%%%%%
%%%%%%%%%%%%%%%%%%%%%%%%%%%%%%%%%%%%%%%%%%%%%%%%%%%%%%%%%%%%%%%%%%%%%%%%%%%%%%%

\section{Lexical rules}
\nodiff

%%%%%%%%%%%%%%%%%%%%%%%%%%%%%%%%%%%%%%%%%%%%%%%%%%%%%%%%%%%%%%%%%%%%%%%%%%%%%%%
%%%%%%%%%%%%%%%%%%%%%%%%%%%%%%%%%%%%%%%%%%%%%%%%%%%%%%%%%%%%%%%%%%%%%%%%%%%%%%%
%%%%%%%%%%%%%%%%%%%%%%%%%%%%%%%%%%%%%%%%%%%%%%%%%%%%%%%%%%%%%%%%%%%%%%%%%%%%%%%

\section{Logic expressions}
\label{sec:expressions}

\except{guarded quantificatication}.

More precisely, grammars of terms and binders presented respectively
Figures~\ref{fig:gram:term} and~\ref{fig:gram:binders} are the same than the one
of \acsl, while Figure~\ref{fig:gram:pred} presents grammar of predicates.
\begin{figure}[htbp]
  \begin{cadre}
    \input{term_modern.bnf}
  \end{cadre}
  \caption{Grammar of terms}
\label{fig:gram:term}
\end{figure}
\begin{figure}[htbp]
  \begin{cadre}
    \input{predicate_modern.bnf}
  \end{cadre}
  \caption{Grammar of predicates}
\label{fig:gram:pred}
\end{figure}
\begin{figure}[htbp]
  \fbox{\begin{minipage}{0.97\textwidth} \input{binders_modern.bnf}
    \end{minipage}}
  \caption{Grammar of binders and type expressions}
\label{fig:gram:binders}
\end{figure}
The only difference between \eacsl and \acsl predicates are
quantifications.

\begin{description}
\item[Quantification] Universal quantification is denoted by
\begin{lstlisting}
\forall $\tau$ $x_1$,$\ldots$,$x_n$;
  $a_1$ <= $x_1$ <= $b_1$ $\ldots$ && $a_n$ <= $x_n$ <= $b_n$
  ==> e
\end{lstlisting} and existential quantification by
\begin{lstlisting}
\exists $\tau$ $x_1$,$\ldots$,$x_n$;
  $a_1$ <= $x_1$ <= $b_1$ $\ldots$ && $a_n$ <= $x_n$ <= $b_n$
  ==> e
\end{lstlisting}
Therefore each quantified variable belongs to a finite constant interval. By
this way, since operator \lstinline|<=| is only valid for integers and reals,
quantification is limited to both of these types (and their corresponding
subtypes). For example, quantification over \C pointers or logic types is not
allowed.
\end{description}

%%%%%%%%%%%%%%%%%%%%%%%%%%%%%%%%%%%%%%%%%%%%%%%%%%%%%%%%%%%%%%%%%%%%%%%%%%%%%%%

\subsection{Operators precedence}
\nodiff

Figure~\ref{fig:precedence} summarizes operator precedences.
\begin{figure}[htbp]
  \begin{center}
    \begin{tabular}{|l|l|l|}
      \hline
      class 	& associativity & operators \\
      \hline
      selection & left & \lstinline|[$\cdots$]| \lstinline|->| \lstinline|.| \\
      unary 	& right & \lstinline|!| \lstinline|~| \lstinline|+|
      \lstinline|-| \lstinline|*| \lstinline|&| \lstinline|(cast)|
      \lstinline|sizeof| \\
      multiplicative & left & \lstinline|*| \lstinline|/|  \lstinline|%| \\
      additive & left & \lstinline|+| \lstinline|-| \\
      shift 	& left & \lstinline|<<| \lstinline|>>| \\
      comparison & left & \lstinline|<| \lstinline|<=| \lstinline|>| \lstinline|>=| \\
      comparison & left & \lstinline|==| \lstinline|!=| \\
      bitwise and & left & \lstinline|&| \\
      bitwise xor & left & \lstinline|^| \\
      bitwise or & left & \lstinline+|+ \\
      bitwise implies & left & \lstinline+-->+ \\
      bitwise equiv & left & \lstinline+<-->+ \\
      connective and     & left & \lstinline|&&| \\
      connective xor & left & \lstinline+^^+ \\
      connective or & left & \lstinline+||+ \\
      connective implies & right & \lstinline|==>| \\
      connective equiv & left & \lstinline|<==>| \\
      ternary connective & right & \lstinline|$\cdots$?$\cdots$:$\cdots$| \\
      binding & left & \Forall{} \Exists{} \Let{} \\
      naming & right & \lstinline|:| \\
      \hline
    \end{tabular}
  \end{center}
  \caption{Operator precedence}
\label{fig:precedence}
\end{figure}


%%%%%%%%%%%%%%%%%%%%%%%%%%%%%%%%%%%%%%%%%%%%%%%%%%%%%%%%%%%%%%%%%%%%%%%%%%%%%%%

\subsection{Semantics}
\label{sec:twovaluedlogic}

\except{undefinedness}

More precisely, while \acsl is a 2-valued logic with only total
functions, \eacsl is a 3-valued logic with partial functions since
expressions may be ``undefined''. In this logic, the semantics of any
predicate containing a C expression which would lead to a runtime
error or whose evaluation does not terminate is undefined. Each tool
implementing \eacsl semantics may do what it wants when interpreting
an undefined predicate. For instance, it may consider the expression
is valid or it may fail. This behavior is consistent with both
\acsl~\cite{acsl} and mainstream specification languages for runtime
assertion checking like \jml~\cite{jml}. Consistency means that, if it
exists and is defined, the \eacsl predicate corresponding to a valid
(resp. invalid) \acsl or \jml predicate is either valid
(resp. invalid). Thus it is possible to reuse tools interpreting \acsl
in order to interpret \eacsl, and it is also possible to perform
runtime assertion checking of \eacsl predicate.

\begin{example}
The semantics of all the predicates below are undefined:
\begin{itemize}
\item \lstinline|1/0 == 1/0|
\item \lstinline|\false && 1/0 == 1/0|
\item \lstinline|1/0 == 1/0 && \false|
\item \lstinline|\forall integer x, -1 <= 1/x <= 1|
\item \lstinline|f(1/0)| for any logic function \lstinline|f|
\item \lstinline|cyclic == cyclic| if \lstinline|cyclic| is a cyclic \C
  structure
\end{itemize}
\end{example}

Reader interested by the implications (especially issues)
of such a choice may read articles of Patrice Chalin~\cite{chalin05,chalin07}.

%%%%%%%%%%%%%%%%%%%%%%%%%%%%%%%%%%%%%%%%%%%%%%%%%%%%%%%%%%%%%%%%%%%%%%%%%%%%%%%

\subsection{Typing}
\nodiff

%%%%%%%%%%%%%%%%%%%%%%%%%%%%%%%%%%%%%%%%%%%%%%%%%%%%%%%%%%%%%%%%%%%%%%%%%%%%%%%

\subsection{Integer arithmetic and machine integers}
\nodiff

%%%%%%%%%%%%%%%%%%%%%%%%%%%%%%%%%%%%%%%%%%%%%%%%%%%%%%%%%%%%%%%%%%%%%%%%%%%%%%%

\subsection{Real numbers and floating point numbers}
\nodiff

%%%%%%%%%%%%%%%%%%%%%%%%%%%%%%%%%%%%%%%%%%%%%%%%%%%%%%%%%%%%%%%%%%%%%%%%%%%%%%%

\subsection{C arrays and pointers}
\nodiff

%%%%%%%%%%%%%%%%%%%%%%%%%%%%%%%%%%%%%%%%%%%%%%%%%%%%%%%%%%%%%%%%%%%%%%%%%%%%%%%

\subsection{Structures, Unions and Arrays in logic}
\nodiff

%%%%%%%%%%%%%%%%%%%%%%%%%%%%%%%%%%%%%%%%%%%%%%%%%%%%%%%%%%%%%%%%%%%%%%%%%%%%%%%

\subsection{String literals}
\nodiff

%%%%%%%%%%%%%%%%%%%%%%%%%%%%%%%%%%%%%%%%%%%%%%%%%%%%%%%%%%%%%%%%%%%%%%%%%%%%%%%
%%%%%%%%%%%%%%%%%%%%%%%%%%%%%%%%%%%%%%%%%%%%%%%%%%%%%%%%%%%%%%%%%%%%%%%%%%%%%%%
%%%%%%%%%%%%%%%%%%%%%%%%%%%%%%%%%%%%%%%%%%%%%%%%%%%%%%%%%%%%%%%%%%%%%%%%%%%%%%%

\section{Function contracts}
\label{sec:fn-behavior}
\index{function contract}\index{contract}

\except{no assigns, terminates and abrupt clauses}

Figure~\ref{fig:gram:contracts} shows grammar of function
contracts. This is a simplified version of \acsl one with no
\lstinline|assigns|, \lstinline|terminates| and \lstinline|abrupt|
clauses. All the other constructs are unchanged.
\begin{figure}[htbp]
  \begin{cadre}
      \input{fn_behavior_modern.bnf}
   \end{cadre}
    \caption{Grammar of function contracts}
  \label{fig:gram:contracts}
\end{figure}

%%%%%%%%%%%%%%%%%%%%%%%%%%%%%%%%%%%%%%%%%%%%%%%%%%%%%%%%%%%%%%%%%%%%%%%%%%%%%%%

\subsection{Built-in constructs %
  \texorpdfstring{\old}{\textbackslash{}old} %
 and \texorpdfstring{\result}{\textbackslash{}result}}

\nodiff

Figure~\ref{fig:gram:oldandresult} summarizes grammar extension of terms with
\lstinline|\old| and \lstinline|\result|.
\begin{figure}[htbp]
  \begin{cadre}
      \input{oldandresult_modern.bnf}
    \end{cadre}
    \caption{\protect\old and \protect\result in terms}
  \label{fig:gram:oldandresult}
\end{figure}


%%%%%%%%%%%%%%%%%%%%%%%%%%%%%%%%%%%%%%%%%%%%%%%%%%%%%%%%%%%%%%%%%%%%%%%%%%%%%%%

\subsection{Simple function contracts}
\label{sec:simplecontracts}

\except{no \lstinline|assigns| clause}

Simple function contracts are now limited to \lstinline|requires| and
\lstinline|ensures| clauses since there is no assigns. Their semantics are
unchanged.

%%%%%%%%%%%%%%%%%%%%%%%%%%%%%%%%%%%%%%%%%%%%%%%%%%%%%%%%%%%%%%%%%%%%%%%%%%%%%%%

\subsection{Contracts with named behaviors}
\label{subsec:behaviors}
\index{function behavior}\index{behavior}

\except{no \lstinline|assigns| clause}

Contracts with named behaviors are now limited to \lstinline|assumes|,
\lstinline|requires| and \lstinline|ensures| clauses since there is no
assigns. Their semantics are unchanched.

%%%%%%%%%%%%%%%%%%%%%%%%%%%%%%%%%%%%%%%%%%%%%%%%%%%%%%%%%%%%%%%%%%%%%%%%%%%%%%%

\subsection{Memory locations and sets of terms}
\label{sec:locations}

\except{limited range and set comprehension}

Figure~\ref{fig:gram:locations} describes grammar of sets of terms. The only
differences with \acsl are that both lower and upper bounds of ranges are
mandatory and that the predicate inside set comprehension must be guarded. In
that way, each set of terms is finite.
\begin{figure}[htbp]
  \fbox{\begin{minipage}{0.97\textwidth}
      \input{loc_modern.bnf}
    \end{minipage}}
  \caption{Grammar for sets of terms}
\label{fig:gram:locations}
\end{figure}

%%%%%%%%%%%%%%%%%%%%%%%%%%%%%%%%%%%%%%%%%%%%%%%%%%%%%%%%%%%%%%%%%%%%%%%%%%%%%%%

\subsection{Default contracts, multiple contracts}
\nodiff

%%%%%%%%%%%%%%%%%%%%%%%%%%%%%%%%%%%%%%%%%%%%%%%%%%%%%%%%%%%%%%%%%%%%%%%%%%%%%%%
%%%%%%%%%%%%%%%%%%%%%%%%%%%%%%%%%%%%%%%%%%%%%%%%%%%%%%%%%%%%%%%%%%%%%%%%%%%%%%%
%%%%%%%%%%%%%%%%%%%%%%%%%%%%%%%%%%%%%%%%%%%%%%%%%%%%%%%%%%%%%%%%%%%%%%%%%%%%%%%

\section{Statement annotations}
\index{annotation}

%%%%%%%%%%%%%%%%%%%%%%%%%%%%%%%%%%%%%%%%%%%%%%%%%%%%%%%%%%%%%%%%%%%%%%%%%%%%%%%

\subsection{Assertions}
\label{sec:assertions}
\indextt{assert}
\nodiff

Figure~\ref{fig:gram:assertions} summarizes grammar for assertions.
\begin{figure}[htbp]
  \begin{cadre}
    \input{assertions_modern.bnf}
  \end{cadre}
  \caption{Grammar for assertions}
  \label{fig:gram:assertions}
\end{figure}

%%%%%%%%%%%%%%%%%%%%%%%%%%%%%%%%%%%%%%%%%%%%%%%%%%%%%%%%%%%%%%%%%%%%%%%%%%%%%%%

\subsection{Loop annotations}
\label{sec:loop_annot}

\except{limited to loop variant}

Figure~\ref{fig:gram:loops} shows grammar for loop annotations. The only
possible loop annotation is loop variant. There is neither
\lstinline|loop assigns| nor \lstinline|loop invariant| clause. About loop
invariant, you have to use \lstinline|assertion| clause instead (see
Section~\ref{sec:assertions}), even if there is no way to encode the inductive
scheme of a loop invariant.
\begin{figure}[htbp]
  \begin{cadre}
    \input{loops_modern.bnf}
  \end{cadre}
  \caption{Grammar for loop annotations}
  \label{fig:gram:loops}
\end{figure}

%%%%%%%%%%%%%%%%%%%%%%%%%%%%%%%%%%%%%%%%%%%%%%%%%%%%%%%%%%%%%%%%%%%%%%%%%%%%%%%

\subsection{Built-in construct \texorpdfstring{\at}{\textbackslash{}at}}
\label{sec:at}
\nodiff

%%%%%%%%%%%%%%%%%%%%%%%%%%%%%%%%%%%%%%%%%%%%%%%%%%%%%%%%%%%%%%%%%%%%%%%%%%%%%%%

\subsection{Statement contracts}
\label{sec:statement_contract}
\index{statement contract}\index{contract}

\except{no \lstinline|assigns| and \lstinline|abrupt| clauses}

Figure~\ref{fig:gram:contracts} shows grammar of statement contracts. Like
function contracts, this is a simplified version of \acsl with neither
\lstinline|assigns| nor \lstinline|abrupt| clauses. 
All the other constructs are unchanged.

\begin{figure}[htbp]
  \begin{cadre}
    \input{st_contracts_modern.bnf}
  \end{cadre}
  \caption{Grammar for statement contracts}
  \label{fig:gram:stcontracts}
\end{figure}

%%%%%%%%%%%%%%%%%%%%%%%%%%%%%%%%%%%%%%%%%%%%%%%%%%%%%%%%%%%%%%%%%%%%%%%%%%%%%%%
%%%%%%%%%%%%%%%%%%%%%%%%%%%%%%%%%%%%%%%%%%%%%%%%%%%%%%%%%%%%%%%%%%%%%%%%%%%%%%%
%%%%%%%%%%%%%%%%%%%%%%%%%%%%%%%%%%%%%%%%%%%%%%%%%%%%%%%%%%%%%%%%%%%%%%%%%%%%%%%

\section{Termination}
\label{sec:termination}
\index{termination}

\except{no general measures and \lstinline|terminates| clauses}

Termination can only be guaranteed by attaching an integer measure to each loop
and each recursive function.

%%%%%%%%%%%%%%%%%%%%%%%%%%%%%%%%%%%%%%%%%%%%%%%%%%%%%%%%%%%%%%%%%%%%%%%%%%%%%%%

\subsection{Integer measures}
\label{sec:integermeasures}
\indexttbs{decreases}\indexttbs{variant}
\nodiff

%%%%%%%%%%%%%%%%%%%%%%%%%%%%%%%%%%%%%%%%%%%%%%%%%%%%%%%%%%%%%%%%%%%%%%%%%%%%%%%

\subsection{General measures}
\absent

%%%%%%%%%%%%%%%%%%%%%%%%%%%%%%%%%%%%%%%%%%%%%%%%%%%%%%%%%%%%%%%%%%%%%%%%%%%%%%%

\subsection{Recursive function calls}
\nodiff

%%%%%%%%%%%%%%%%%%%%%%%%%%%%%%%%%%%%%%%%%%%%%%%%%%%%%%%%%%%%%%%%%%%%%%%%%%%%%%%

\subsection{Non-terminating functions}
\absent

%%%%%%%%%%%%%%%%%%%%%%%%%%%%%%%%%%%%%%%%%%%%%%%%%%%%%%%%%%%%%%%%%%%%%%%%%%%%%%%
%%%%%%%%%%%%%%%%%%%%%%%%%%%%%%%%%%%%%%%%%%%%%%%%%%%%%%%%%%%%%%%%%%%%%%%%%%%%%%%
%%%%%%%%%%%%%%%%%%%%%%%%%%%%%%%%%%%%%%%%%%%%%%%%%%%%%%%%%%%%%%%%%%%%%%%%%%%%%%%

\section{Logic specifications}
\label{sec:logicspec}
\index{logic specification}\index{specification}

\except{neither polymorphism nor lemma}

Figure~\ref{fig:gram:logic} presents grammar of logic definitions. This is the
same than the one of \acsl without polymorphic definitions nor lemmas.

\begin{figure}[htbp]
  \fbox{\begin{minipage}{0.97\linewidth}\vfill \input{logic_modern.bnf}
    \vfill\end{minipage}}
  \caption{Grammar for global logic definitions}
\label{fig:gram:logic}
\end{figure}

%%%%%%%%%%%%%%%%%%%%%%%%%%%%%%%%%%%%%%%%%%%%%%%%%%%%%%%%%%%%%%%%%%%%%%%%%%%%%%%

\subsection{Predicate and function definitions}
\nodiff

%%%%%%%%%%%%%%%%%%%%%%%%%%%%%%%%%%%%%%%%%%%%%%%%%%%%%%%%%%%%%%%%%%%%%%%%%%%%%%%

\subsection{Lemmas}
\absent

%%%%%%%%%%%%%%%%%%%%%%%%%%%%%%%%%%%%%%%%%%%%%%%%%%%%%%%%%%%%%%%%%%%%%%%%%%%%%%%

\subsection{Inductive predicates}
\absent

%%%%%%%%%%%%%%%%%%%%%%%%%%%%%%%%%%%%%%%%%%%%%%%%%%%%%%%%%%%%%%%%%%%%%%%%%%%%%%%

\subsection{Axiomatic definitions}
\absent

%%%%%%%%%%%%%%%%%%%%%%%%%%%%%%%%%%%%%%%%%%%%%%%%%%%%%%%%%%%%%%%%%%%%%%%%%%%%%%%

\subsection{Polymorphic logic types}
\absent

%%%%%%%%%%%%%%%%%%%%%%%%%%%%%%%%%%%%%%%%%%%%%%%%%%%%%%%%%%%%%%%%%%%%%%%%%%%%%%%

\subsection{Recursive logic definitions}
\index{recursion}
\nodiff

%%%%%%%%%%%%%%%%%%%%%%%%%%%%%%%%%%%%%%%%%%%%%%%%%%%%%%%%%%%%%%%%%%%%%%%%%%%%%%%

\subsection{Higher-order logic constructions}
\absent

%%%%%%%%%%%%%%%%%%%%%%%%%%%%%%%%%%%%%%%%%%%%%%%%%%%%%%%%%%%%%%%%%%%%%%%%%%%%%%%

\subsection{Concrete logic types}
\absent

%%%%%%%%%%%%%%%%%%%%%%%%%%%%%%%%%%%%%%%%%%%%%%%%%%%%%%%%%%%%%%%%%%%%%%%%%%%%%%%

\subsection{Hybrid functions and predicates}
\label{sec:logicalstates}
\index{hybrid!function}
\index{hybrid!predicate}
\nodiff

%%%%%%%%%%%%%%%%%%%%%%%%%%%%%%%%%%%%%%%%%%%%%%%%%%%%%%%%%%%%%%%%%%%%%%%%%%%%%%%

\subsection{Memory footprint specification: \texorpdfstring{\lstinline|reads|}{reads} clause}
\absent

%%%%%%%%%%%%%%%%%%%%%%%%%%%%%%%%%%%%%%%%%%%%%%%%%%%%%%%%%%%%%%%%%%%%%%%%%%%%%%%

\subsection{Specification Modules}
\label{sec:specmodules}
\absent

%%%%%%%%%%%%%%%%%%%%%%%%%%%%%%%%%%%%%%%%%%%%%%%%%%%%%%%%%%%%%%%%%%%%%%%%%%%%%%%
%%%%%%%%%%%%%%%%%%%%%%%%%%%%%%%%%%%%%%%%%%%%%%%%%%%%%%%%%%%%%%%%%%%%%%%%%%%%%%%
%%%%%%%%%%%%%%%%%%%%%%%%%%%%%%%%%%%%%%%%%%%%%%%%%%%%%%%%%%%%%%%%%%%%%%%%%%%%%%%

\section{Pointers and physical adressing}
\label{sec:pointers}

There is mostly no special built-in \eacsl construct dealing with pointers and
physical adressing.

%%%%%%%%%%%%%%%%%%%%%%%%%%%%%%%%%%%%%%%%%%%%%%%%%%%%%%%%%%%%%%%%%%%%%%%%%%%%%%%

\subsection{Memory blocks and pointer dereferencing}
\label{subsec:memory}
% \absentexcept{\lstinline|\null|} % JS: doesn't work, don't know why
\emph{No such feature in \eacsl, but } \lstinline|\null|\emph{.}

The only existing built-in construct dealing with memory state is
\lstinline|\null|.  In particular, \lstinline|\valid| does not exist: you can
only check for nullity and therefore it is not possible to check whether
dereferencing a non null pointer is safe.

%%%%%%%%%%%%%%%%%%%%%%%%%%%%%%%%%%%%%%%%%%%%%%%%%%%%%%%%%%%%%%%%%%%%%%%%%%%%%%%

\subsection{Separation}
\absent

%%%%%%%%%%%%%%%%%%%%%%%%%%%%%%%%%%%%%%%%%%%%%%%%%%%%%%%%%%%%%%%%%%%%%%%%%%%%%%%

\subsection{Allocation and deallocation}
\absent

%%%%%%%%%%%%%%%%%%%%%%%%%%%%%%%%%%%%%%%%%%%%%%%%%%%%%%%%%%%%%%%%%%%%%%%%%%%%%%%
%%%%%%%%%%%%%%%%%%%%%%%%%%%%%%%%%%%%%%%%%%%%%%%%%%%%%%%%%%%%%%%%%%%%%%%%%%%%%%%
%%%%%%%%%%%%%%%%%%%%%%%%%%%%%%%%%%%%%%%%%%%%%%%%%%%%%%%%%%%%%%%%%%%%%%%%%%%%%%%

\section{Sets as first-class values}
\index{location}
\nodiff

%%%%%%%%%%%%%%%%%%%%%%%%%%%%%%%%%%%%%%%%%%%%%%%%%%%%%%%%%%%%%%%%%%%%%%%%%%%%%%%
%%%%%%%%%%%%%%%%%%%%%%%%%%%%%%%%%%%%%%%%%%%%%%%%%%%%%%%%%%%%%%%%%%%%%%%%%%%%%%%
%%%%%%%%%%%%%%%%%%%%%%%%%%%%%%%%%%%%%%%%%%%%%%%%%%%%%%%%%%%%%%%%%%%%%%%%%%%%%%%

\section{Abrupt termination}
\absent

%%%%%%%%%%%%%%%%%%%%%%%%%%%%%%%%%%%%%%%%%%%%%%%%%%%%%%%%%%%%%%%%%%%%%%%%%%%%%%%
%%%%%%%%%%%%%%%%%%%%%%%%%%%%%%%%%%%%%%%%%%%%%%%%%%%%%%%%%%%%%%%%%%%%%%%%%%%%%%%
%%%%%%%%%%%%%%%%%%%%%%%%%%%%%%%%%%%%%%%%%%%%%%%%%%%%%%%%%%%%%%%%%%%%%%%%%%%%%%%

\section{Dependencies information}
\absent

%%%%%%%%%%%%%%%%%%%%%%%%%%%%%%%%%%%%%%%%%%%%%%%%%%%%%%%%%%%%%%%%%%%%%%%%%%%%%%%
%%%%%%%%%%%%%%%%%%%%%%%%%%%%%%%%%%%%%%%%%%%%%%%%%%%%%%%%%%%%%%%%%%%%%%%%%%%%%%%
%%%%%%%%%%%%%%%%%%%%%%%%%%%%%%%%%%%%%%%%%%%%%%%%%%%%%%%%%%%%%%%%%%%%%%%%%%%%%%%

\section{Data invariants}
\label{sec:invariants}
\index{data invariant}\index{global invariant}\index{type invariant}
\index{invariant!data}\index{invariant!global}\index{invariant!type}

\nodiff

Figure~\ref{fig:gram:datainvariants} summarizes grammar for declarations of data
invariants.
\begin{figure}[htbp]
  \fbox{\begin{minipage}{0.97\linewidth}
      \input{data_invariants_modern.bnf}
    \end{minipage}}
  \caption{Grammar for declarations of data invariants}
\label{fig:gram:datainvariants}
\end{figure}

%%%%%%%%%%%%%%%%%%%%%%%%%%%%%%%%%%%%%%%%%%%%%%%%%%%%%%%%%%%%%%%%%%%%%%%%%%%%%%%

\subsection{Semantics}
\nodiff

%%%%%%%%%%%%%%%%%%%%%%%%%%%%%%%%%%%%%%%%%%%%%%%%%%%%%%%%%%%%%%%%%%%%%%%%%%%%%%%

\subsection{Model variables and model fields}
\index{model}

\nodiff

Figure~\ref{fig:gram:model} summarizes grammar for declarations of model
variables and fields.
\begin{figure}[htbp]
  \fbox{\begin{minipage}{0.97\linewidth}
      \input{model_modern.bnf}
    \end{minipage}}
  \caption{Grammar for declarations of model variables and fields}
\label{fig:gram:model}
\end{figure}

%%%%%%%%%%%%%%%%%%%%%%%%%%%%%%%%%%%%%%%%%%%%%%%%%%%%%%%%%%%%%%%%%%%%%%%%%%%%%%%
%%%%%%%%%%%%%%%%%%%%%%%%%%%%%%%%%%%%%%%%%%%%%%%%%%%%%%%%%%%%%%%%%%%%%%%%%%%%%%%
%%%%%%%%%%%%%%%%%%%%%%%%%%%%%%%%%%%%%%%%%%%%%%%%%%%%%%%%%%%%%%%%%%%%%%%%%%%%%%%

\section{Ghost variables and statements}
\label{sec:ghost}
\index{ghost}

\except{no specific construct for volatile variables}

Figure~\ref{fig:gram:ghost} summarizes grammar for ghost statements which is the
same than the one of \acsl.
\begin{figure}[htbp]
  \fbox{\begin{minipage}{0.98\linewidth}
      \input{ghost_modern.bnf}
    \end{minipage}}
  \caption{Grammar for ghost statements}
\label{fig:gram:ghost}
\end{figure}


%%%%%%%%%%%%%%%%%%%%%%%%%%%%%%%%%%%%%%%%%%%%%%%%%%%%%%%%%%%%%%%%%%%%%%%%%%%%%%%

\subsection{Volatile variables}
\absent

%%%%%%%%%%%%%%%%%%%%%%%%%%%%%%%%%%%%%%%%%%%%%%%%%%%%%%%%%%%%%%%%%%%%%%%%%%%%%%%
%%%%%%%%%%%%%%%%%%%%%%%%%%%%%%%%%%%%%%%%%%%%%%%%%%%%%%%%%%%%%%%%%%%%%%%%%%%%%%%
%%%%%%%%%%%%%%%%%%%%%%%%%%%%%%%%%%%%%%%%%%%%%%%%%%%%%%%%%%%%%%%%%%%%%%%%%%%%%%%

\section{Undefined values, dangling pointers}
\absent
